\section{math}

\subsection{知識}
\begin{small}
\begin{itemize}
    \item Euler のトーシェント関数:
    $n$ の相異なる素因数を $p_1,p_2,\dots$ として,
    \[\varphi(n)=n\prod_{p_i} \frac{p_i-1}{p_i}\]

    \item Möbius 関数:
    \[\mu(n)=\begin{cases} 0 & \text{if $n$ has a square divisor} \\ (-1)^k & \text{if $n$ has $k$ prime factors} \end{cases}\]

    \item Monmort 数 (撹乱順列の個数):
    \[W_1=0,\,W_2=1,\,W_k=(k-1)(W_{k-1}+W_{k-2})\]

    \item 第1種 Stirling 数 $s(n,k)$
    \begin{itemize}
        \item 定義: \[x(x-1)\cdots(x-(n-1)) = \sum_{k=0}^n s(n,k) x^k\]
        \item $s(n,k)$ の絶対値 (${n \brack k}$ と書く) は,$n$ 要素の置換のうち,$k$ 個のサイクルに分解されるものの個数である.
        \item 漸化式: \[{n\brack k} = {n-1\brack k-1} + (n-1){n-1 \brack k}\]
    \end{itemize}

    \item 第2種 Stirling 数 ${n \brace k}$
    \begin{itemize}
        \item 定義: \[{n \brace k} = \frac{1}{k!} \sum_{i=0}^n (-1)^{k-i} \binom{k}{i} i^n\]
        \item ${n \brace k}$ は,$n$ 個の区別できるボールを, $k$ 個の区別できない箱に,すべての箱に1つ以上のボールが入るように分配する方法の数である.
        \item 漸化式: \[{n\brace k} = {n-1\brace k-1} + k{n-1 \brace k}\]
    \end{itemize}

    \item Lagrange 補間:
    $(x_1,y_1),\dots,(x_n,y_n)$ を通る $n+1$ 次多項式は,
    \[\delta_i(x)=\frac{(x-x_1)\cdots(x-x_{i-1})(x-x_{i+1})\cdot(x_i-x_n)}{(x_i-x_1)\cdots(x_i-x_{i-1})(x_i-x_{i+1})\cdot(x_i-x_n)}\]
    として,
    \[\sum_i y_i \delta_i(x)\]
\end{itemize}
\end{small}

\subsection{行列}

\begin{small}
\begin{itemize}
    \item 掃き出し法: 以下のコードを参照
    \item 行列式: 掃き出して対角要素の積.swap のたびに $-1$ をかけることに注意
    \item 逆行列: $\begin{pmatrix}A & I\end{pmatrix}$ を掃き出す
    \item 連立一次方程式: $\begin{pmatrix}A & \boldsymbol{b}\end{pmatrix}$ を掃き出す
\end{itemize}
\end{small}

\begin{lstlisting}
int pivot = 0;
for (int j = 0; j < n; ++j) {
    int i = pivot;
    while (i < m && eq(A[i][j], T(0))) ++i;
    if (i == m) continue;

    if (i != pivot) A[i].swap(A[pivot]);

    T p = A[pivot][j];
    for (int l = j; l < n; ++l) A[pivot][l] /= p;

    for (int k = 0; k < m; ++k) {
        if (k == pivot) continue;
        T v = A[k][j];
        for (int l = j; l < n; ++l) {
            A[k][l] -= A[pivot][l] * v;
        }
    }

    ++pivot;
}
return A;
\end{lstlisting}

\subsection{商が一定の区間の列挙}

\begin{lstlisting}
ll i = 1;
while (i <= n) {
    ll q = n / i;
    ll j = n / q + 1;
    // [i, j) では n/k の値が一定
    i = j;
}
\end{lstlisting}

\subsection{Garner's Algorithm}

\begin{small}
\begin{markdown}
連立合同式 $x \equiv b\_i \mod m\_i \quad (i=1,\dots,n)$ の解

- `long long garner(vector<long long> b, vector<long long> m, long long mod)`
    - 連立合同式を満たす最小の非負整数を法$mod$で求める.
    - 時間計算量: $O(n^2)$

\end{markdown}
\end{small}

\begin{lstlisting}
#include "extgcd.cpp"

long long garner(const std::vector<long long>& b, std::vector<long long> m, long long mod) {
    m.push_back(mod);
    int n = m.size();
    std::vector<long long> coeffs(n, 1);
    std::vector<long long> consts(n, 0);
    for (int k = 0; k < n - 1; ++k) {
        long long t = (b[k] - consts[k]) * mod_inv(coeffs[k], m[k]) % m[k];
        if (t < 0) t += m[k];
        for (int i = k + 1; i < n; ++i) {
            consts[i] = (consts[i] + t * coeffs[i]) % m[i];
            coeffs[i] = coeffs[i] * m[k] % m[i];
        }
    }
    return consts.back();
}
\end{lstlisting}

\subsection{Extended Euclidean Algorithm}

\begin{small}
\begin{markdown}
$ax + by = \gcd(a, b)$ の解 $(x, y)$ を1組求める

- `pair<long long, long long> extgcd(long long a, long long b)`
    - $ax + by = \gcd(a, b)$ の解 $(x, y)$ を1組求める
    - 時間計算量: $O(\log \min(a, b))$
- `long long mod\_inv(long long a, long long mod)`
    - $a$ の法$mod$での逆元を求める
\end{markdown}
\end{small}

\begin{lstlisting}
std::pair<long long, long long> extgcd(long long a, long long b) {
    long long s = a, sx = 1, sy = 0, t = b, tx = 0, ty = 1;
    while (t) {
        long long q = s / t;
        std::swap(s -= t * q, t);
        std::swap(sx -= tx * q, tx);
        std::swap(sy -= ty * q, ty);
    }
    return {sx, sy};
}

long long mod_inv(long long a, long long mod) {
    long long inv = extgcd(a, mod).first;
    return (inv % mod + mod) % mod;
}
\end{lstlisting}

\subsection{Fast Prime Number Algorithms}

\begin{small}
\begin{markdown}
- `bool is\_prime(long long n)`
    - $n$ を素数判定する
    - 時間計算量: $O(\log^3 n)$
- `vector<long long> prime\_factor(long long n)`
    - $n$ の素因数のリストを返す
    - 時間計算量: $\mathrm{expected}\ O(n^{\frac{1}{4}} \log n)$
\end{markdown}
\end{small}

\begin{lstlisting}
namespace fast_prime {

class LargeModint {
    using mint = LargeModint;

public:
    static long long& get_mod() noexcept {
        static long long mod = 1;
        return mod;
    }

    static void set_mod(long long mod) {
        get_mod() = mod;
    }

    LargeModint(long long y = 0) noexcept : x(y >= 0 ? y % get_mod() : (y % get_mod() + get_mod()) % get_mod()) {}

    long long value() const noexcept { return x; }

    mint& operator+=(const mint& r) noexcept { if ((x += r.x) >= get_mod()) x -= get_mod(); return *this; }
    mint& operator-=(const mint& r) noexcept { if ((x += get_mod() - r.x) >= get_mod()) x -= get_mod(); return *this; }
    mint& operator*=(const mint& r) noexcept { x = static_cast<long long>((__int128_t) x * r.x % get_mod()); return *this; }

    mint operator-() const noexcept { return mint(-x); }

    mint operator+(const mint& r) const noexcept { return mint(*this) += r; }
    mint operator-(const mint& r) const noexcept { return mint(*this) -= r; }
    mint operator*(const mint& r) const noexcept { return mint(*this) *= r; }

    bool operator==(const mint& r) const noexcept { return x == r.x; }
    bool operator!=(const mint& r) const noexcept { return x != r.x; }

    mint pow(long long n) const noexcept {
        mint ret(1), mul(x);
        while (n > 0) {
            if (n & 1) ret *= mul;
            mul *= mul;
            n >>= 1;
        }
        return ret;
    }

private:
    long long x;
};

using mint = LargeModint;

bool is_prime(long long n) {
    if (n == 2) return true;
    if (n == 1 || n % 2 == 0) return false;

    mint::set_mod(n);
    int s = 0;
    long long d = n - 1;
    while (!(d & 1)) d >>= 1, ++s;
    // https://miller-rabin.appspot.com/
    for (mint a : {2, 325, 9375, 28178, 450775, 9780504, 1795265022}) {
        if (a == 0) break;
        mint y = a.pow(d);
        if (y == 1) continue;
        bool probably_prime = false;
        for (int r = 0; r < s; ++r) {
            if (y == n - 1) {
                probably_prime = true;
                break;
            }
            y *= y;
        }
        if (!probably_prime) return false;
    }
    return true;
}

unsigned long long randll(long long lb, long long ub) {
    static std::random_device rd;
    static std::mt19937_64 rng(rd());
    std::uniform_int_distribution<long long> rand(lb, ub - 1);
    return rand(rng);
}

long long pollards_rho(long long n) {
    if (n % 2 == 0) return 2;
    if (is_prime(n)) return n;

    mint::set_mod(n);
    while (true) {
        mint x = randll(2, n);
        mint y = x;
        mint c = randll(1, n);
        long long d = 1;
        while (d == 1) {
            x = x * x + c;
            y = y * y + c;
            y = y * y + c;
            d = std::gcd((x - y).value(), n);
        }
        if (d < n) return d;
    }
}

std::vector<long long> prime_factor(long long n) {
    if (n <= 1) return {};
    long long p = pollards_rho(n);
    if (p == n) return {p};
    auto l = prime_factor(p);
    auto r = prime_factor(n / p);
    std::copy(r.begin(), r.end(), std::back_inserter(l));
    return l;
}

} // namespace fast_prime
\end{lstlisting}

\subsection{Modular Arithmetic}

\begin{small}
\begin{markdown}
- `long long mod\_pow(long long a, long long e, int mod)`
    - $a^e \mod mod$ を計算する
    - 時間計算量: $O(\log e)$
- `long long mod\_inv(long long a, int mod)`
    - $a$ の $\mod mod$ での逆元を計算する
    - 時間計算量: $O(\log mod)$
- `int mod\_log(long long a, long long b, int mod)`
    - $a^x \equiv b \mod mod$ を満たす $x$ を求める.存在しない場合は $-1$ を返す.
    - 時間計算量: $O(\sqrt{mod})$
- `vector<int> mod\_inv\_table(int n, int mod)`
  - $1, 2, \dots, n$ の $\mathrm{mod}\,mod$ での逆元を計算する.
  - 時間計算量: $O(n)$
\end{markdown}
\end{small}

\begin{lstlisting}
#include "euler_totient.cpp"

/*
 * Modular Exponentiation
 */
long long mod_pow(long long a, long long e, int mod) {
    long long ret = 1;
    while (e > 0) {
        if (e & 1) ret = ret * a % mod;
        a = a * a % mod;
        e >>= 1;
    }
    return ret;
}

long long mod_inv(long long a, int mod) {
    return mod_pow(a, mod - 2, mod);
}

/*
 * Discrete Logarithm
 */
int mod_log(long long a, long long b, int mod) {
    // make a and mod coprime
    a %= mod;
    b %= mod;
    long long k = 1, add = 0, g;
    while ((g = std::gcd(a, mod)) > 1) {
        if (b == k) return add;
        if (b % g) return -1;
        b /= g;
        mod /= g;
        ++add;
        k = k * a / g % mod;
    }

    // baby-step
    const int m = sqrt(mod) + 1;
    std::unordered_map<long long, int> baby_index;
    long long baby = b;
    for (int i = 0; i <= m; ++i) {
        baby_index[baby] = i;
        baby = baby * a % mod;
    }

    // giant-step
    long long am = 1;
    for (int i = 0; i < m; ++i) am = am * a % mod;
    long long giant = k;
    for (int i = 1; i <= m; ++i) {
        giant = giant * am % mod;
        if (baby_index.count(giant)) {
            return i * m - baby_index[giant] + add;
        }
    }
    return -1;
}

/**
 * Table of Modular Inverses
 */
std::vector<int> mod_inv_table(int n, int mod) {
    std::vector<int> inv(n + 1, 1);
    for (int i = 2; i <= n; ++i) {
        inv[i] = mod - 1LL * inv[mod % i] * (mod / i) % mod;
    }
    return inv;
}
\end{lstlisting}

\subsection{Sum of Floor of Linear}

\begin{small}
\begin{markdown}
一次関数の床関数の和 $\sum\_{i=0}^{N-1} \left\lfloor \frac{Ai + B}{M} \right\rfloor$

- `long long floor\_sum(long long n, long long m, long long a, long long b)`
    - $\sum\_{i=0}^{N-1} \left\lfloor \frac{Ai + B}{M} \right\rfloor$ を計算する
\end{markdown}
\end{small}

\begin{lstlisting}
long long floor_sum(long long n, long long m, long long a, long long b) {
    long long sum = 0;
    if (a >= m) {
        sum += (a / m) * n * (n - 1) / 2;
        a %= m;
    }
    if (b >= m) {
        sum += (b / m) * n;
        b %= m;
    }
    long long y = (a * n + b) / m;
    if (y == 0) return sum;
    long long x = (m * y - b + a - 1) / a;
    sum += (n - x) * y + floor_sum(y, a, m, a * x - m * y + b);
    return sum;
}
\end{lstlisting}

\subsection{Polynomial}

\begin{small}
\begin{markdown}
**Description**

係数が`Modint`または`ArbitraryModint`である多項式を扱う.

空間計算量: $O(n)$

**Operations**

- `Polynomial inv(int deg)`
    - $\frac{1}{f(x)}$ を $deg - 1$ 次の項まで計算する
    - 時間計算量: $O(n \log n)$
- `Polynomial exp(int deg)`
    - $\exp(f(x))$ を $deg - 1$ 次の項まで計算する
    - 時間計算量: $O(n \log n)$
- `Polynomial log(int deg)`
    - $\log(f(x))$ を $deg - 1$ 次の項まで計算する
    - 時間計算量: $O(n \log n)$
- `Polynomial pow(long long k, int deg)`
    - $(f(x))^k$ を $deg - 1$ 次の項まで計算する
    - 時間計算量: $O(n \log n)$
- `Polynomial diff()`
    - $f'(x)$ を計算する
    - 時間計算量: $O(n)$
- `Polynomial integral()`
    - $\int f(x)$ を計算する
    - 時間計算量: $O(n)$
- `Polynomial taylor\_shift(long long c)`
    - $f(x+c)$ を計算する
    - 時間計算量: $O(n \log n)$

\end{markdown}
\end{small}

\begin{lstlisting}
#include "../convolution/ntt.hpp"

template <typename mint>
class Polynomial : public std::vector<mint> {
    using Poly = Polynomial;

   public:
    using std::vector<mint>::vector;
    using std::vector<mint>::operator=;

    Poly pre(int size) const {
        return Poly(this->begin(),
                    this->begin() + std::min((int)this->size(), size));
    }

    Poly rev(int deg = -1) const {
        auto ret = *this;
        if (deg != -1) ret.resize(deg, 0);
        return Poly(ret.rbegin(), ret.rend());
    }

    void trim() {
        while (!this->empty() && this->back() == 0) this->pop_back();
    }

    // --- unary operation ---

    Poly& operator-() const {
        auto ret = *this;
        for (auto& x : ret) x = -x;
        return ret;
    }

    // -- binary operation with constant

    Poly& operator+=(const mint& rhs) {
        if (this->empty()) this->resize(1);
        (*this)[0] += rhs;
        return *this;
    }

    Poly& operator-=(const mint& rhs) {
        if (this->empty()) this->resize(1);
        (*this)[0] -= rhs;
        return *this;
    }

    Poly& operator*=(const mint& rhs) {
        for (auto& x : *this) x *= rhs;
        return *this;
    }

    Poly& operator/=(const mint& rhs) { return *this *= rhs.inv(); }

    Poly operator+(const mint& rhs) const { return Poly(*this) += rhs; }
    Poly operator-(const mint& rhs) const { return Poly(*this) -= rhs; }
    Poly operator*(const mint& rhs) const { return Poly(*this) *= rhs; }
    Poly operator/(const mint& rhs) const { return Poly(*this) /= rhs; }

    // --- binary operation with polynomial ---

    Poly& operator+=(const Poly& rhs) {
        if (this->size() < rhs.size()) this->resize(rhs.size());
        for (int i = 0; i < (int)rhs.size(); ++i) (*this)[i] += rhs[i];
        return *this;
    }

    Poly& operator-=(const Poly& rhs) {
        if (this->size() < rhs.size()) this->resize(rhs.size());
        for (int i = 0; i < (int)rhs.size(); ++i) (*this)[i] -= rhs[i];
        return *this;
    }

    Poly& operator*=(const Poly& rhs) {
        *this = convolution(*this, rhs);
        return *this;
        // // naive convolution O(N^2)
        // std::vector<mint> res(this->size() + rhs.size() - 1);
        // for (int i = 0; i < (int)this->size(); ++i) {
        //     for (int j = 0; j < (int)rhs.size(); ++j) {
        //         res[i + j] += (*this)[i] * rhs[j];
        //     }
        // }
        // return *this = res;
    }

    Poly& operator/=(const Poly& rhs) {
        if (this->size() < rhs.size()) {
            this->clear();
            return *this;
        }
        int n = this->size() - rhs.size() + 1;
        return *this = (rev().pre(n) * rhs.rev().inv(n)).pre(n).rev(n);
    }

    Poly& operator%=(const Poly& rhs) {
        *this -= *this / rhs * rhs;
        trim();
        return *this;
    }

    std::pair<Poly, Poly> divmod(const Poly& rhs) {
        auto q = *this / rhs;
        auto r = *this - q * rhs;
        r.trim();
        return {q, r};
    }

    Poly operator+(const Poly& rhs) const { return Poly(*this) += rhs; }
    Poly operator-(const Poly& rhs) const { return Poly(*this) -= rhs; }
    Poly operator*(const Poly& rhs) const { return Poly(*this) *= rhs; }
    Poly operator/(const Poly& rhs) const { return Poly(*this) /= rhs; }
    Poly operator%(const Poly& rhs) const { return Poly(*this) %= rhs; }

    // --- shift operation ---

    Poly operator<<(int n) const {
        auto ret = *this;
        ret.insert(ret.begin(), n, 0);
        return ret;
    }

    Poly operator>>(int n) const {
        if ((int)this->size() <= n) return {};
        auto ret = *this;
        ret.erase(ret.begin(), ret.begin() + n);
        return ret;
    }

    // --- evaluation ---

    mint operator()(const mint& x) {
        mint y = 0, powx = 1;
        for (int i = 0; i < (int)this->size(); ++i) {
            for (auto c : *this) {
                y += c * powx;
                powx *= x;
            }
            return y;
        }
    }

    // --- other operations ---

    Poly inv(int deg = -1) const {
        assert((*this)[0] != mint(0));
        if (deg == -1) deg = this->size();
        Poly res = {(*this)[0].inv()};
        for (int d = 1; d < deg; d <<= 1) {
            auto f = pre(2 * d);
            auto g = res;
            f.resize(2 * d);
            g.resize(2 * d);

            // g_{n+1} = g_n * (2 - g_n * f) mod x^{2^{n+1}}

            ntt(f);
            ntt(g);
            for (int i = 0; i < 2 * d; ++i) f[i] *= g[i];
            intt(f);

            for (int i = 0; i < d; ++i) f[i] = 0;

            ntt(f);
            for (int i = 0; i < 2 * d; ++i) f[i] *= g[i];
            intt(f);

            res.resize(2 * d);
            auto coef = mint(2 * d).inv().pow(2);
            for (int i = d; i < 2 * d; ++i) res[i] = -f[i] * coef;
        }
        return res.pre(deg);
    }

    Poly exp(int deg = -1) const {
        assert((*this)[0] == mint(0));
        if (deg == -1) deg = this->size();
        Poly ret = {mint(1)};
        for (int i = 1; i < deg; i <<= 1) {
            ret = (ret * (this->pre(i << 1) + mint(1) - ret.log(i << 1)))
                      .pre(i << 1);
        }
        return ret;
    }

    Poly log(int deg = -1) const {
        assert((*this)[0] == mint(1));
        if (deg == -1) deg = this->size();
        return (diff() * inv(deg)).pre(deg - 1).integral();
    }

    Poly pow(long long k, int deg = -1) const {
        if (k == 0) return {1};
        if (deg == -1) deg = this->size();
        auto ret = *this;
        int cnt0 = 0;
        while (cnt0 < (int)ret.size() && ret[cnt0] == 0) ++cnt0;
        if (cnt0 > (deg - 1) / k) return {};
        ret = ret >> cnt0;
        deg -= cnt0 * k;
        ret = ((ret / ret[0]).log(deg) * k).exp(deg) * ret[0].pow(k);
        ret = ret << (cnt0 * k);
        return ret;
    }

    Poly diff() const {
        Poly ret(std::max(0, (int)this->size() - 1));
        for (int i = 1; i <= (int)ret.size(); ++i)
            ret[i - 1] = (*this)[i] * mint(i);
        return ret;
    }

    Poly integral() const {
        Poly ret(this->size() + 1);
        ret[0] = mint(0);
        for (int i = 0; i < (int)ret.size() - 1; ++i)
            ret[i + 1] = (*this)[i] / mint(i + 1);
        return ret;
    }

    Poly taylor_shift(long long c) const {
        const int n = this->size();
        std::vector<mint> fact(n, 1), fact_inv(n, 1);
        for (int i = 1; i < n; ++i) fact[i] = fact[i - 1] * i;
        fact_inv[n - 1] = mint(1) / fact[n - 1];
        for (int i = n - 1; i > 0; --i) fact_inv[i - 1] = fact_inv[i] * i;

        auto ret = *this;
        Poly e(n + 1);
        e[0] = 1;
        mint p = c;
        for (int i = 1; i < n; ++i) {
            ret[i] *= fact[i];
            e[i] = p * fact_inv[i];
            p *= c;
        }
        ret = (ret.rev() * e).pre(n).rev();
        for (int i = n - 1; i >= 0; --i) {
            ret[i] *= fact_inv[i];
        }
        return ret;
    }
};

\end{lstlisting}