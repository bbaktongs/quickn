\usepackage[a4paper,left=7mm,right=7mm,bottom=7mm,top=12mm,headsep=0.5mm]{geometry}  % 余白調整
\usepackage{fancyhdr}  % ヘッダー・フッターの書式
\usepackage{listings,jlisting}  % コードブロック
\usepackage{color}  % 文字に色をつける
\usepackage{amsmath,amssymb}  % 数式
\usepackage{lastpage}  % 総ページ数の取得
\usepackage[hybrid]{markdown} % markdown 埋め込み

% ヘッダー・フッターの設定
\pagestyle{fancy}
\fancyhf{}
\rhead{\thepage/\pageref{LastPage}}  % ヘッダー右
\lhead{suzukaze\_Aobayama}  % ヘッダー左
% \rfoot{}  % フッター右

% コードブロックの設定
\lstset{
  language={C++},
  backgroundcolor={\color[RGB]{250,250,250}}, % 背景色
  basicstyle={\ttfamily\footnotesize}, % 標準の書式. フォントサイズなど
  identifierstyle={}, % キーワード以外の書式
  commentstyle={\itshape \color[RGB]{0,128,0}},% コメントの書式
  keywordstyle={\bfseries \color{blue}},% キーワードの書式
  otherkeywords={constexpr}, % キーワードの追加
  stringstyle={\ttfamily \color[RGB]{163,21,21}}, % 文字列の書式
  directivestyle={\color[RGB]{128,128,128}},
  showstringspaces=false, % 文字列中の空白の表示
  frame=single, % 枠の追加
  breaklines=true, % 自動改行
  columns=fixed,%
  basewidth=0.5em,
  xrightmargin=0zw, %
  xleftmargin=1zw,%
  numbers=left, % 行番号表示位置
  numberstyle={\scriptsize}, % 行番号の書式
  stepnumber=1, % 行番号の表示行数間隔
  numbersep=1zw, % 行番号と本文の間隔
  lineskip=-0.6ex, % 行間隔
  tabsize=4,
}
