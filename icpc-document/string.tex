\section{string}

\subsection{Rolling Hash}

\begin{small}
\begin{markdown}
$\mod 2^{61} - 1$

- `RollingHash(string s, long long base)`
- `RollingHash(vector<T> s, long long base)`
    - `s` のハッシュ値を計算する
    - 時間計算量: $O(n)$
- `static long long generate\_base()`
    - ランダムな基数を返す
    - 時間計算量: $O(1)$
- `long long query(int l, int r)`
    - 区間 $[l, r)$ のハッシュ値を返す
    - 時間計算量: $O(1)$
- `long long combine(long long h1, long long h2, int len2)`
    - ハッシュ値 $h1$ と $h2$ を結合する.$h2$ の長さを $len2$ である
    - 時間計算量: $O(1)$
- `void push\_back(char c)`
    - 文字 $c$ を末尾に結合する
    - 時間計算量: $O(1)$

\end{markdown}
\end{small}

\begin{lstlisting}
class RollingHash {
public:
    static long long generate_base() {
        std::random_device rd;
        std::mt19937_64 rng(rd());
        std::uniform_int_distribution<long long> rand(1, mod - 1);
        return rand(rng);
    }

    RollingHash() = default;
    RollingHash(const std::string& s, long long base) : RollingHash(std::vector<char>(s.begin(), s.end()), base) {}
    template <typename T>
    RollingHash(const std::vector<T>& s, long long base)
        : base(base), hashed(s.size() + 1), power(s.size() + 1) {
        power[0] = 1;
        for (int i = 0; i < (int) s.size(); ++i) {
            power[i + 1] = mul(power[i], base);
            hashed[i + 1] = add(mul(hashed[i], base), s[i]);
        }
    }

    long long query(int l, int r) const {
        return add(hashed[r], mod - mul(hashed[l], power[r - l]));
    }

    long long combine(long long h1, long long h2, int len2) const {
        return add(mul(h1, power[len2]), h2);
    }

    void push_back(char c) {
        power.push_back(mul(power.back(), base));
        hashed.push_back(add(mul(hashed.back(), base), c));
    }

private:
    static constexpr long long mod = (1LL << 61) - 1;

    static inline long long add(long long a, long long b) {
        if ((a += b) >= mod) a -= mod;
        return a;
    }

    static inline long long mul(long long a, long long b) {
        __int128_t c = (__int128_t) a * b;
        return add(c >> 61, c & mod);
    }

    const long long base;
    std::vector<long long> hashed, power;
};
\end{lstlisting}

\subsection{Suffix Array}

\begin{lstlisting}
template <typename T>
std::vector<int> suffix_array(const std::vector<T>& s) {
    int n = s.size();
    std::vector<int> sa(n);
    std::iota(sa.begin(), sa.end(), 0);
    std::sort(sa.begin(), sa.end(), [&](int i, int j) {
        return s[i] < s[j];
    });
    int cl = 0;
    std::vector<int> rank(n);
    for (int i = 1; i < n; ++i) {
        if (s[sa[i - 1]] != s[sa[i]]) ++cl;
        rank[sa[i]] = cl;
    }
    std::vector<int> tmp(n), nrank(n), cnt(n);
    for (int k = 1; k < n; k <<= 1) {
        // sort by second half
        int cnt1 = 0, cnt2 = k;
        for (int i = 0; i < n; ++i) {
            int j = sa[i] - k;
            if (j >= 0) tmp[cnt2++] = j;
            else tmp[cnt1++] = j + n;
        }

        // sort by first half
        std::fill(cnt.begin(), cnt.end(), 0);
        for (int i = 0; i < n; ++i) ++cnt[rank[tmp[i]]];
        for (int i = 1; i < n; ++i) cnt[i] += cnt[i - 1];
        for (int i = n - 1; i >= 0; --i) sa[--cnt[rank[tmp[i]]]] = tmp[i];

        // assign new rank
        nrank[sa[0]] = 0;
        cl = 0;
        for (int i = 1; i < n; ++i) {
            if (rank[sa[i - 1]] != rank[sa[i]]
                || (sa[i - 1] + k < n ? rank[sa[i - 1] + k] : -1) != (sa[i] + k < n ? rank[sa[i] + k] : -1)) {
                ++cl;
            }
            nrank[sa[i]] = cl;
        }
        std::swap(rank, nrank);
    }
    return sa;
}
\end{lstlisting}

\subsection{Longest Common Prefix Array}

\begin{small}
\begin{markdown}
`lcp[i]` は接尾辞 `s[sa[i]..]` と接尾辞 `s[sa[i + 1]..]` の先頭で共通している文字数
\end{markdown}
\end{small}

\begin{lstlisting}
template <typename T>
std::vector<int> lcp_array(const std::vector<T>& s,
                           const std::vector<int>& sa) {
    int n = s.size();
    std::vector<int> rank(n);
    for (int i = 0; i < n; ++i) rank[sa[i]] = i;
    int h = 0;
    std::vector<int> lcp(n - 1);
    for (int i = 0; i < n; ++i) {
        if (h > 0) --h;
        if (rank[i] == 0) continue;
        int j = sa[rank[i] - 1];
        while (j + h < n && i + h < n && s[j + h] == s[i + h]) ++h;
        lcp[rank[i] - 1] = h;
    }
    return lcp;
}
\end{lstlisting}

\subsection{Z Array}

\begin{small}
\begin{markdown}
文字列 `S` と `S[i:]` の最長共通接頭辞の長さ
\end{markdown}
\end{small}

\begin{lstlisting}
std::vector<int> z_array(const std::string& s) {
    int n = s.size();
    std::vector<int> z(n);
    z[0] = n;
    int l = 0, r = 0;
    for (int i = 1; i < n; ++i) {
        int k = i - l;
        if (i <= r && z[k] < r - i + 1) {
            z[i] = z[k];
        } else {
            l = i;
            if (i > r) r = i;
            while (r < n && s[r - l] == s[r]) ++r;
            --r;
            z[i] = r - l + 1;
        }
    }
    return z;
}
\end{lstlisting}

\subsection{Aho-Corasick Algorithm}

\begin{small}
\begin{markdown}
- `void insert(string p)`
    - パターン $p$ を挿入する
    - 時間計算量: $O(\vert p\vert)$
- `void build()`
    - オートマトンを構築する
    - 時間計算量: $O(\sum \vert p\vert)$
- `int get\_next(int i, char c)`
    - 状態 $i$ にいるときに文字 $c$ が出現したときの遷移先の状態を返す
- `long long count(string s)`
    - 文字列 $s$ に対する各パターンのマッチ回数の合計を返す
    - 時間計算量: $O(\vert s\vert + \sum \vert p\vert)$
- `vector<pair<int, int>> match(string s)`
    - 文字列 $s$ に対する各パターンのマッチ位置を返す
    - 時間計算量: $O(\vert s\vert + \sum \vert p\vert)$
\end{markdown}
\end{small}

\begin{lstlisting}
class AhoCorasick {
public:
    struct Node {
        std::map<char, int> ch;
        std::vector<int> accept;
        int link = -1;
        int cnt = 0;

        Node() = default;
    };

    std::vector<Node> states;
    std::map<int, int> accept_state;

    explicit AhoCorasick() : states(1) {}

    void insert(const std::string& s, int id = -1) {
        int i = 0;
        for (char c : s) {
            if (!states[i].ch.count(c)) {
                states[i].ch[c] = states.size();
                states.emplace_back();
            }
            i = states[i].ch[c];
        }
        ++states[i].cnt;
        states[i].accept.push_back(id);
        accept_state[id] = i;
    }

    void clear() {
        states.clear();
        states.emplace_back();
    }

    int get_next(int i, char c) const {
        while (i != -1 && !states[i].ch.count(c)) i = states[i].link;
        return i != -1 ? states[i].ch.at(c) : 0;
    }

    void build() {
        std::queue<int> que;
        que.push(0);
        while (!que.empty()) {
            int i = que.front();
            que.pop();

            for (auto [c, j] : states[i].ch) {
                states[j].link = get_next(states[i].link, c);
                states[j].cnt += states[states[j].link].cnt;

                auto& a = states[j].accept;
                auto& b = states[states[j].link].accept;
                std::vector<int> accept;
                std::set_union(a.begin(), a.end(), b.begin(), b.end(), std::back_inserter(accept));
                a = accept;

                que.push(j);
            }
        }
    }

    long long count(const std::string& str) const {
        long long ret = 0;
        int i = 0;
        for (auto c : str) {
            i = get_next(i, c);
            ret += states[i].cnt;
        }
        return ret;
    }

    // list of (id, index)
    std::vector<std::pair<int, int>> match(const std::string& str) const {
        std::vector<std::pair<int, int>> ret;
        int i = 0;
        for (int k = 0; k < (int) str.size(); ++k) {
            char c = str[k];
            i = get_next(i, c);
            for (auto id : states[i].accept) {
                ret.emplace_back(id, k);
            }
        }
        return ret;
    }
};

\end{lstlisting}

\subsection{Knuth-Morris-Pratt Algorithm}

\begin{small}
\begin{markdown}
prefix function (`P[:i+1]` の接尾辞でもある最長の proper prefix)

**Operations**

- `vector<int> prefix\_function(vector<T> s)`
- `vector<int> prefix\_function(string s)`
    - 数列 `s` の prefix function を計算する
    - 時間計算量: $O(n)$
- `vector<int> kmp(vector<T> txt, vector<T> pat, vector<int> pf)`
- `vector<int> kmp(string txt, string pat, vector<int> pf)`
    - `txt`中の`pat`の出現位置を列挙する
    - 時間計算量: $O(n + m)$
- `vector<vector<pair<int, bool>>> matching\_automaton(string s)`
    - `s` のマッチングオートマトンを返す.`ret[i][c]` は,$i$ 文字マッチしているときに文字 `c` が出現したときの遷移先と,全体がマッチしたか否かを返す.
    - 時間計算量: $O(nA)$, $A$ はアルファベットサイズ

\end{markdown}
\end{small}

\begin{lstlisting}
template <typename T>
std::vector<int> prefix_function(const std::vector<T>& s) {
    const int n = s.size();
    std::vector<int> ret(n);
    int len = 0;
    for (int i = 1; i < n; ++i) {
        if (s[i] == s[len]) {
            ++len;
            ret[i] = len;
        } else {
            if (len != 0) {
                len = ret[len - 1];
                --i;
            } else {
                ret[i] = 0;
            }
        }
    }
    return ret;
}

template <typename T>
std::vector<int> kmp(const std::vector<T>& txt, const std::vector<T>& pat, const std::vector<int>& pf) {
    int n = txt.size(), m = pat.size();
    std::vector<int> match;
    int i = 0, j = 0;
    while (i < n) {
        if (pat[j] == txt[i]) {
            ++i;
            ++j;
        }
        if (j == m) {
            match.push_back(i - j);
            j = pf[j - 1];
        } else if (i < n && pat[j] != txt[i]) {
            if (j != 0) {
                j = pf[j - 1];
            } else {
                ++i;
            }
        }
    }
    return match;
}

std::vector<int> prefix_function(const std::string& s) {
    return prefix_function(std::vector<char>(s.begin(), s.end()));
}

std::vector<int> kmp(const std::string& txt, const std::string& pat, const std::vector<int>& pf) {
    return kmp(std::vector<char>(txt.begin(), txt.end()), std::vector<char>(pat.begin(), pat.end()), pf);
}

template <int AlphabetSize, int Offset>
std::vector<std::vector<std::pair<int, bool>>> matching_automaton(const std::string& s) {
    const int n = s.size();
    auto lps = prefix_function(s);
    std::vector<std::vector<std::pair<int, bool>>> aut(n, std::vector<std::pair<int, bool>>(AlphabetSize));
    for (int i = 0; i < n; ++i) {
        for (int c = 0; c < AlphabetSize; ++c) {
            if (Offset + c == s[i]) {
                if (i == n - 1) aut[i][c] = {lps[i], true};
                else aut[i][c] = {i + 1, false};
            } else {
                aut[i][c] = {i > 0 ? aut[lps[i - 1]][c].first : 0, 0};
            }
        }
    }
    return aut;
}
\end{lstlisting}

\subsection{Trie}

\begin{small}
\begin{markdown}
- `void insert(string s, int id)`
    - 文字列 $s$ を挿入する
    - 時間計算量: $O(\vert s\vert)$
- `void compress()`
    - トライ木を圧縮して Patricia trie を構築する.これにより,木の木の深さが $O(\sqrt{\sum \vert s \vert})$ になる.
    - 時間計算量: $O(\sum \vert s\vert)$
\end{markdown}
\end{small}

\begin{lstlisting}
class Trie {
public:
    Trie() : root(std::make_shared<Node>()) {}

    void insert(const std::string& s, int id) { insert(root, s, id, 0); }

    void compress() { compress(root); }

protected:
    struct Node;
    using node_ptr = std::shared_ptr<Node>;

    struct Node {
        std::map<char, node_ptr> ch;
        std::vector<int> accept;
        int sz = 0;
        node_ptr par;
        std::string str;

        Node() = default;
    };

    const node_ptr root;

    void insert(const node_ptr& t, const std::string& s, int id, int k) {
        ++t->sz;
        if (k == (int) s.size()) {
            t->accept.push_back(id);
            return;
        }
        int c = s[k];
        if (!t->ch.count(c)) {
            t->ch[c] = std::make_shared<Node>();
            t->ch[c]->par = t;
            t->ch[c]->str = c;
        }
        insert(t->ch[c], s, id, k + 1);
    }

    void compress(node_ptr t) {
        while (t->accept.empty() && t->ch.size() == 1) {
            auto u = t->ch.begin()->second;
            t->ch = u->ch;
            t->accept = u->accept;
            t->str += u->str;
            for (auto [c, w] : t->ch) w->par = t;
            compress(t);
        }
        for (auto [c, u] : t->ch) {
            compress(u);
        }
    }
};
\end{lstlisting}