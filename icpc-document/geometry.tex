\section{geometry}

\subsection{Geometry}

\begin{small}
\begin{markdown}
**Description**

幾何アルゴリズム詰め合わせ

`Vec` は `std::complex<T>` のエイリアスである.

できるだけ整数だけの計算も扱えるようにした

**Operations**

時間計算量は明示しない限り $O(1)$.

** `geometry.hpp`**

- `T dot(Vec a, Vec b)`
    - 内積を計算する

- `T cross(Vec a, Vec b)`
    - 外積の $z$ 座標を計算する

- `Vec rot(Vec a, T ang)`
    - $a$ を角 $ang$ だけ回転させる

- `Vec perp(Vec a)`
    - $a$ を角 $\pi/2$ だけ回転させる

- `Vec projection(Line l, Vec p)`
    - 点 $p$ の直線 $l$ 上の射影を求める

- `Vec reflection(Line l, Vec p)`
    - 点 $p$ の直線 $l$ に関して対称な点を求める

- `int ccw(Vec a, Vec b, Vec c)`
    - $a,b,c$ が同一直線上にあるなら0, $a \rightarrow b \rightarrow c$ が反時計回りなら1,そうでなければ-1を返す

- `void sort\_by\_arg(vector<Vec> pts)`
  - 与えられた点を偏角ソートする (ソート順は[この問題](https://judge.yosupo.jp/problem/sort\_points\_by\_argument)に準拠)
  - 時間計算量: $O(n\log n)$

** `intersect.hpp`**

- `bool intersect(Segment s, Vec p)`

  `int intersect(Polygon poly, Vec p)`

  `bool intersect(Segment s, Segment t)`

  `bool intersect(Polygon poly1, Polygon poly2)`

  `int intersect(Circle c1, Circle c2)`
    - 引数で与えられた2つの対象が交差するか判定する.詳細な仕様はコードのコメントを参照

** `dist.hpp`**

 - `T dist(Line l, Vec p)`

   `T dist(Segment s, Vec p)`

   `T dist(Segment s, Segment t)`
    - 引数で与えられた2つの対象の距離を計算する

** `intersection.hpp`**

- `Vec intersection(Line l, Line m)`

- `vector<Vec> intersection(Circle c, Line l)`

  `vector<Vec> intersection(Circle c1, Circle c2)`
    - 引数で与えられた2つの対象の交点を返す

- `T area\_intersection(Circle c1, Circle c2)`
    - 円 $c\_1,c\_2$ の共通部分の面積を求める

** `bisector.hpp`**

- `Line bisector(Segment s)`
    - 線分 $s$ の垂直二等分線を返す

- `pair<Line, Line> bisector(Line l, Line m)`
    - 直線 $l,m$ の角二等分線を返す.$l,m$ が平行なら,両者の中間の線を返す (つまり,$l,m$ から等距離にある点の集合を返す)

** `triangle.hpp`**

- `Vec centroid(Vec A, Vec B, Vec C)`
    - 三角形 $ABC$ の重心を返す
    - ***NOT VERIFIED***

- `Vec incenter(Vec A, Vec B, Vec C)`
    - 三角形 $ABC$ の内心を返す

- `Vec circumcenter(Vec A, Vec B, Vec C)`
    - 三角形 $ABC$ の外心を返す

** `tangent.hpp`**

- `pair<Vec, Vec> tangent\_points(Circle c, Vec p)`
    - 点 $p$ を通り円 $c$ に接する接線と $c$ の接点を返す

- `vector<Line> common\_tangents(Circle c1, Circle c2)`
    - 円 $c\_1,c\_2$ の共通接線を返す

** `polygon.hpp`**

- `T area(Polygon poly)`
    - 多角形 $poly$ の面積を求める
    - 時間計算量: $O(n)$

- `T is\_convex(Polygon poly)`
    - 多角形 $poly$ が凸か判定する.`poly` は反時計回りに与えられる必要がある
    - 時間計算量: $O(n)$

- `Polygon convex\_cut(Polygon poly, Line l)`
    - 多角形 $poly$ を直線 $l$ で切断する.詳細な仕様は [凸多角形の切断](https://onlinejudge.u-aizu.ac.jp/courses/library/4/CGL/4/CGL\_4\_C) を参照.
    - 時間計算量: $O(n)$

- `Polygon halfplane\_intersection(vector<pair<Vec, Vec>> hps)`
    - 半平面の集合が与えられたとき,それらの共通部分 (凸多角形になる) を返す.半平面は, $\{\boldsymbol{x}\mid(\boldsymbol{x}-\boldsymbol{p})\cdot \boldsymbol{n}\geq 0\}$ で表したときに $(\boldsymbol{n},\boldsymbol{p})$ の形で与える.
    - 時間計算量: $O(n\log n)$

- `class PolygonContainment`
    - 多角形と点の包含関係に関するクエリを処理するデータ構造
    - 時間計算量: $O(\log n)$ per query

\end{markdown}
\end{small}

\begin{lstlisting}
// note that if T is of an integer type, std::abs does not work
using T = double;
using Vec = std::complex<T>;

std::istream& operator>>(std::istream& is, Vec& p) {
    T x, y;
    is >> x >> y;
    p = {x, y};
    return is;
}

T dot(const Vec& a, const Vec& b) { return (std::conj(a) * b).real(); }

T cross(const Vec& a, const Vec& b) { return (std::conj(a) * b).imag(); }

const T PI = std::acos(-1);
constexpr T eps = 1e-10;
inline bool eq(T a, T b) { return std::abs(a - b) <= eps; }
inline bool eq(Vec a, Vec b) { return std::abs(a - b) <= eps; }
inline bool lt(T a, T b) { return a < b - eps; }
inline bool leq(T a, T b) { return a <= b + eps; }

struct Line {
    Vec p1, p2;
    Line() = default;
    Line(const Vec& p1, const Vec& p2) : p1(p1), p2(p2) {}
    Vec dir() const { return p2 - p1; }
};

struct Segment : Line {
    using Line::Line;
};

struct Circle {
    Vec c;
    T r;
    Circle() = default;
    Circle(const Vec& c, T r) : c(c), r(r) {}
};

using Polygon = std::vector<Vec>;

Vec rot(const Vec& a, T ang) { return a * Vec(std::cos(ang), std::sin(ang)); }

Vec perp(const Vec& a) { return Vec(-a.imag(), a.real()); }

Vec projection(const Line& l, const Vec& p) {
    return l.p1 + dot(p - l.p1, l.dir()) * l.dir() / std::norm(l.dir());
}

Vec reflection(const Line& l, const Vec& p) {
    return T(2) * projection(l, p) - p;
}

// 0: collinear
// 1: counter-clockwise
// -1: clockwise
int ccw(const Vec& a, const Vec& b, const Vec& c) {
    if (eq(cross(b - a, c - a), 0)) return 0;
    if (lt(cross(b - a, c - a), 0)) return -1;
    return 1;
}

void sort_by_arg(std::vector<Vec>& pts) {
    std::sort(pts.begin(), pts.end(), [&](auto& p, auto& q) {
        if ((p.imag() < 0) != (q.imag() < 0)) return (p.imag() < 0);
        if (cross(p, q) == 0) {
            if (p == Vec(0, 0))
                return !(q.imag() < 0 || (q.imag() == 0 && q.real() > 0));
            if (q == Vec(0, 0))
                return (p.imag() < 0 || (p.imag() == 0 && p.real() > 0));
            return (p.real() > q.real());
        }
        return (cross(p, q) > 0);
    });
}
\end{lstlisting}

\subsection{intersect.hpp}

\begin{small}
\begin{markdown}

\end{markdown}
\end{small}

\begin{lstlisting}
#include "geometry.hpp"

bool intersect(const Segment& s, const Vec& p) {
    Vec u = s.p1 - p, v = s.p2 - p;
    return eq(cross(u, v), 0) && leq(dot(u, v), 0);
}

// 0: outside
// 1: on the border
// 2: inside
int intersect(const Polygon& poly, const Vec& p) {
    const int n = poly.size();
    bool in = 0;
    for (int i = 0; i < n; ++i) {
        auto a = poly[i] - p, b = poly[(i + 1) % n] - p;
        if (eq(cross(a, b), 0) && (lt(dot(a, b), 0) || eq(dot(a, b), 0)))
            return 1;
        if (a.imag() > b.imag()) std::swap(a, b);
        if (leq(a.imag(), 0) && lt(0, b.imag()) && lt(cross(a, b), 0)) in ^= 1;
    }
    return in ? 2 : 0;
}

int intersect(const Segment& s, const Segment& t) {
    auto a = s.p1, b = s.p2;
    auto c = t.p1, d = t.p2;
    if (ccw(a, b, c) != ccw(a, b, d) && ccw(c, d, a) != ccw(c, d, b)) return 2;
    if (intersect(s, c) || intersect(s, d) || intersect(t, a) ||
        intersect(t, b))
        return 1;
    return 0;
}

// true if they have positive area in common or touch on the border
bool intersect(const Polygon& poly1, const Polygon& poly2) {
    const int n = poly1.size();
    const int m = poly2.size();
    for (int i = 0; i < n; ++i) {
        for (int j = 0; j < m; ++j) {
            if (intersect(Segment(poly1[i], poly1[(i + 1) % n]),
                          Segment(poly2[j], poly2[(j + 1) % m]))) {
                return true;
            }
        }
    }
    return intersect(poly1, poly2[0]) || intersect(poly2, poly1[0]);
}

// 0: inside
// 1: inscribe
// 2: intersect
// 3: circumscribe
// 4: outside
int intersect(const Circle& c1, const Circle& c2) {
    T d = std::abs(c1.c - c2.c);
    if (lt(d, std::abs(c2.r - c1.r))) return 0;
    if (eq(d, std::abs(c2.r - c1.r))) return 1;
    if (eq(c1.r + c2.r, d)) return 3;
    if (lt(c1.r + c2.r, d)) return 4;
    return 2;
}
\end{lstlisting}

\subsection{intersection.hpp}

\begin{small}
\begin{markdown}

\end{markdown}
\end{small}

\begin{lstlisting}
#include "dist.hpp"
#include "geometry.hpp"

Vec intersection(const Line& l, const Line& m) {
    assert(!eq(cross(l.dir(), m.dir()), 0));  // not parallel
    Vec r = m.p1 - l.p1;
    return l.p1 + cross(m.dir(), r) / cross(m.dir(), l.dir()) * l.dir();
}

std::vector<Vec> intersection(const Circle& c, const Line& l) {
    T d = dist(l, c.c);
    if (lt(c.r, d)) return {};  // no intersection
    Vec e1 = l.dir() / std::abs(l.dir());
    Vec e2 = perp(e1);
    if (ccw(c.c, l.p1, l.p2) == 1) e2 *= -1;
    if (eq(c.r, d)) return {c.c + d * e2};  // tangent
    T t = std::sqrt(c.r * c.r - d * d);
    return {c.c + d * e2 + t * e1, c.c + d * e2 - t * e1};
}

std::vector<Vec> intersection(const Circle& c1, const Circle& c2) {
    T d = std::abs(c1.c - c2.c);
    if (lt(c1.r + c2.r, d)) return {};  // outside
    Vec e1 = (c2.c - c1.c) / std::abs(c2.c - c1.c);
    Vec e2 = perp(e1);
    if (lt(d, std::abs(c2.r - c1.r))) return {};                  // contain
    if (eq(d, std::abs(c2.r - c1.r))) return {c1.c + c1.r * e1};  // tangent
    T x = (c1.r * c1.r - c2.r * c2.r + d * d) / (2 * d);
    T y = std::sqrt(c1.r * c1.r - x * x);
    return {c1.c + x * e1 + y * e2, c1.c + x * e1 - y * e2};
}

T area_intersection(const Circle& c1, const Circle& c2) {
    T d = std::abs(c2.c - c1.c);
    if (leq(c1.r + c2.r, d)) return 0;    // outside
    if (leq(d, std::abs(c2.r - c1.r))) {  // inside
        T r = std::min(c1.r, c2.r);
        return PI * r * r;
    }
    T ans = 0;
    T a;
    a = std::acos((c1.r * c1.r + d * d - c2.r * c2.r) / (2 * c1.r * d));
    ans += c1.r * c1.r * (a - std::sin(a) * std::cos(a));
    a = std::acos((c2.r * c2.r + d * d - c1.r * c1.r) / (2 * c2.r * d));
    ans += c2.r * c2.r * (a - std::sin(a) * std::cos(a));
    return ans;
}

\end{lstlisting}

\subsection{dist.hpp}

\begin{small}
\begin{markdown}

\end{markdown}
\end{small}

\begin{lstlisting}
#include "geometry.hpp"
#include "intersect.hpp"

T dist(const Line& l, const Vec& p) {
    return std::abs(cross(p - l.p1, l.dir())) / std::abs(l.dir());
}

T dist(const Segment& s, const Vec& p) {
    if (lt(dot(p - s.p1, s.dir()), 0)) return std::abs(p - s.p1);
    if (lt(dot(p - s.p2, -s.dir()), 0)) return std::abs(p - s.p2);
    return std::abs(cross(p - s.p1, s.dir())) / std::abs(s.dir());
}

T dist(const Segment& s, const Segment& t) {
    if (intersect(s, t)) return T(0);
    return std::min({dist(s, t.p1), dist(s, t.p2), dist(t, s.p1), dist(t, s.p2)});
}

\end{lstlisting}

\subsection{Convex Hull}

\begin{small}
\begin{markdown}
**Description**

与えられた点の凸包を求める.この実装では Graham scan アルゴリズムを用いている.

**Operations**

- `vector<Vec> convex\_hull(vector<Vec> pts)`
    - `pts` の凸包の境界の頂点を返す
    - 時間計算量: $O(n\log n)$
\end{markdown}
\end{small}

\begin{lstlisting}
#include "geometry.hpp"

std::vector<Vec> convex_hull(std::vector<Vec>& pts) {
    int n = pts.size();
    if (n == 1) return pts;
    std::sort(pts.begin(), pts.end(), [](const Vec& v1, const Vec& v2) {
        return (v1.imag() != v2.imag()) ? (v1.imag() < v2.imag()) : (v1.real() < v2.real());
    });
    int k = 0; // the number of vertices in the convex hull
    std::vector<Vec> ch(2 * n);
    // right
    for (int i = 0; i < n; ++i) {
        while (k > 1 && lt(cross(ch[k-1] - ch[k-2], pts[i] - ch[k-1]), 0)) --k;
        ch[k++] = pts[i];
    }
    int t = k;
    // left
    for (int i = n - 2; i >= 0; --i) {
        while (k > t && lt(cross(ch[k-1] - ch[k-2], pts[i] - ch[k-1]), 0)) --k;
        ch[k++] = pts[i];
    }
    ch.resize(k - 1);
    return ch;
}

\end{lstlisting}

\subsection{polygon.hpp}

\begin{small}
\begin{markdown}

\end{markdown}
\end{small}

\begin{lstlisting}
#include "geometry.hpp"
#include "intersection.hpp"

T area(const Polygon& poly) {
    const int n = poly.size();
    T res = 0;
    for (int i = 0; i < n; ++i) {
        res += cross(poly[i], poly[(i + 1) % n]);
    }
    return std::abs(res) / T(2);
}

bool is_convex(const Polygon& poly) {
    int n = poly.size();
    for (int i = 0; i < n; ++i) {
        if (lt(cross(poly[(i + 1) % n] - poly[i],
                     poly[(i + 2) % n] - poly[(i + 1) % n]),
               0)) {
            return false;
        }
    }
    return true;
}

Polygon convex_cut(const Polygon& poly, const Line& l) {
    const int n = poly.size();
    Polygon res;
    for (int i = 0; i < n; ++i) {
        auto p = poly[i], q = poly[(i + 1) % n];
        if (ccw(l.p1, l.p2, p) != -1) {
            if (res.empty() || !eq(res.back(), p)) {
                res.push_back(p);
            }
        }
        if (ccw(l.p1, l.p2, p) * ccw(l.p1, l.p2, q) < 0) {
            auto c = intersection(Line(p, q), l);
            if (res.empty() || !eq(res.back(), c)) {
                res.push_back(c);
            }
        }
    }
    return res;
}

Polygon halfplane_intersection(std::vector<std::pair<Vec, Vec>> hps) {
    using Hp = std::pair<Vec, Vec>;  // (normal vector, a point on the border)

    auto intersection = [&](const Hp& l1, const Hp& l2) -> Vec {
        auto d = l2.second - l1.second;
        return l1.second +
               (dot(d, l2.first) / cross(l1.first, l2.first)) * perp(l1.first);
    };

    // check if the halfplane h contains the point p
    auto contains = [&](const Hp& h, const Vec& p) -> bool {
        return dot(p - h.second, h.first) > 0;
    };

    constexpr T INF = 1e15;
    hps.emplace_back(Vec(1, 0), Vec(-INF, 0));  // -INF <= x
    hps.emplace_back(Vec(-1, 0), Vec(INF, 0));  // x <= INF
    hps.emplace_back(Vec(0, 1), Vec(0, -INF));  // -INF <= y
    hps.emplace_back(Vec(0, -1), Vec(0, INF));  // y <= INF

    std::sort(hps.begin(), hps.end(), [&](const auto& h1, const auto& h2) {
        return std::arg(h1.first) < std::arg(h2.first);
    });

    std::deque<Hp> dq;
    int len = 0;
    for (auto& hp : hps) {
        while (len > 1 &&
               !contains(hp, intersection(dq[len - 1], dq[len - 2]))) {
            dq.pop_back();
            --len;
        }

        while (len > 1 && !contains(hp, intersection(dq[0], dq[1]))) {
            dq.pop_front();
            --len;
        }

        // parallel
        if (len > 0 && eq(cross(dq[len - 1].first, hp.first), 0)) {
            // opposite
            if (lt(dot(dq[len - 1].first, hp.first), 0)) {
                return {};
            }
            // same
            if (!contains(hp, dq[len - 1].second)) {
                dq.pop_back();
                --len;
            } else
                continue;
        }

        dq.push_back(hp);
        ++len;
    }

    while (len > 2 &&
           !contains(dq[0], intersection(dq[len - 1], dq[len - 2]))) {
        dq.pop_back();
        --len;
    }

    while (len > 2 && !contains(dq[len - 1], intersection(dq[0], dq[1]))) {
        dq.pop_front();
        --len;
    }

    if (len < 3) return {};

    std::vector<Vec> poly(len);
    for (int i = 0; i < len - 1; ++i) {
        poly[i] = intersection(dq[i], dq[i + 1]);
    }
    poly[len - 1] = intersection(dq[len - 1], dq[0]);
    return poly;
}

class PolygonContainment {
   public:
    explicit PolygonContainment(Polygon poly) : poly(poly) {}

    // 0: outside
    // 1: on the border
    // 2: inside
    int query(Vec pt) {
        auto c1 = cross(poly[1] - poly[0], pt - poly[0]);
        auto c2 = cross(poly.back() - poly[0], pt - poly[0]);
        if (lt(c1, 0) || lt(0, c2)) return 0;

        int lb = 1, ub = (int)poly.size() - 1;
        while (ub - lb > 1) {
            int m = (lb + ub) / 2;
            if (leq(0, cross(poly[m] - poly[0], pt - poly[0])))
                lb = m;
            else
                ub = m;
        }
        auto c = cross(poly[lb] - pt, poly[ub] - pt);
        if (lt(c, 0)) return 0;
        if (eq(c, 0) || eq(c1, 0) || eq(c2, 0)) return 1;
        return 2;
    }

   private:
    Polygon poly;
};
\end{lstlisting}

\subsection{tangent.hpp}

\begin{small}
\begin{markdown}

\end{markdown}
\end{small}

\begin{lstlisting}
#include "geometry.hpp"
#include "intersect.hpp"
#include "intersection.hpp"

std::pair<Vec, Vec> tangent_points(const Circle& c, const Vec& p) {
    auto m = (p + c.c) / T(2);
    auto is = intersection(c, Circle(m, std::abs(p - m)));
    return {is[0], is[1]};
}

// for each l, l.p1 is a tangent point of c1
std::vector<Line> common_tangents(Circle c1, Circle c2) {
    assert(!eq(c1.c, c2.c) || !eq(c1.r, c2.r));
    int cnt = intersect(c1, c2);  // number of common tangents
    std::vector<Line> ret;
    if (cnt == 0) {
        return ret;
    }

    // external
    if (eq(c1.r, c2.r)) {
        auto d = c2.c - c1.c;
        Vec e(-d.imag(), d.real());
        e = e / std::abs(e) * c1.r;
        ret.push_back(Line(c1.c + e, c1.c + e + d));
        ret.push_back(Line(c1.c - e, c1.c - e + d));
    } else {
        auto p = (-c2.r*c1.c + c1.r*c2.c) / (c1.r - c2.r);
        if (cnt == 1) {
            Vec q(-p.imag(), p.real());
            return {Line(p, q)};
        } else {
            auto [a, b] = tangent_points(c1, p);
            ret.push_back(Line(a, p));
            ret.push_back(Line(b, p));
        }
    }

    // internal
    auto p = (c2.r*c1.c + c1.r*c2.c) / (c1.r + c2.r);
    if (cnt == 3) {
        Vec q(-p.imag(), p.real());
        ret.push_back(Line(p, q));
    } else if (cnt == 4) {
        auto [a, b] = tangent_points(c1, p);
        ret.push_back(Line(a, p));
        ret.push_back(Line(b, p));
    }

    return ret;
}

\end{lstlisting}

\subsection{triangle.hpp}

\begin{small}
\begin{markdown}

\end{markdown}
\end{small}

\begin{lstlisting}
#include "geometry.hpp"
#include "intersection.hpp"
#include "bisector.hpp"

Vec centroid(const Vec& A, const Vec& B, const Vec& C) {
    assert(ccw(A, B, C) != 0);
    return (A + B + C) / T(3);
}

Vec incenter(const Vec& A, const Vec& B, const Vec& C) {
    assert(ccw(A, B, C) != 0);
    T a = std::abs(B - C);
    T b = std::abs(C - A);
    T c = std::abs(A - B);
    return (a*A + b*B + c*C) / (a + b + c);
}

Vec circumcenter(const Vec& A, const Vec& B, const Vec& C) {
    assert(ccw(A, B, C) != 0);
    return intersection(bisector(Segment(A, B)), bisector(Segment(A, C)));
}

// large error but beautiful
// Vec circumcenter(const Vec& A, const Vec& B, const Vec& C) {
//     assert(ccw(A, B, C) != 0);
//     Vec p = C - B, q = A - C, r = B - A;
//     T a = std::norm(p) * dot(q, r);
//     T b = std::norm(q) * dot(r, p);
//     T c = std::norm(r) * dot(p, q);
//     return (a*A + b*B + c*C) / (a + b + c);
// }

\end{lstlisting}

\subsection{bisector.hpp}

\begin{small}
\begin{markdown}

\end{markdown}
\end{small}

\begin{lstlisting}
#include "geometry.hpp"
#include "intersection.hpp"

Line bisector(const Segment& s) {
    auto m = (s.p1 + s.p2) / T(2);
    return Line(m, m + Vec(-s.dir().imag(), s.dir().real()));
}

std::pair<Line, Line> bisector(const Line& l, const Line& m) {
    // parallel
    if (eq(cross(l.dir(), m.dir()), 0)) {
        auto n = Line(l.p1, l.p1 + perp(l.dir()));
        auto p = intersection(n, m);
        auto m = (l.p1 + p) / T(2);
        return {Line(m, m + l.dir()), Line()};
    }
    auto p = intersection(l, m);
    T ang = (std::arg(l.dir()) + std::arg(m.dir())) / T(2);
    auto b1 = Line(p, p + std::polar(T(1), ang));
    auto b2 = Line(p, p + std::polar(T(1), ang + PI / T(2)));
    return {b1, b2};
}

\end{lstlisting}

\subsection{geometry3d.hpp}

\begin{small}
\begin{markdown}

\end{markdown}
\end{small}

\begin{lstlisting}
/**
 * @brief 3D Geometry
 */
// following functions from the 2d library also work for 3d without
// any modification:
// projection, reflection, dist, centroid, incenter

using T = double;

struct Vec {
    T x, y, z;
    Vec() = default;
    constexpr Vec(T x, T y, T z) : x(x), y(y), z(z) {}
    constexpr Vec& operator+=(const Vec& r) {
        x += r.x;
        y += r.y;
        z += r.z;
        return *this;
    }
    constexpr Vec& operator-=(const Vec& r) {
        x -= r.x;
        y -= r.y;
        z -= r.z;
        return *this;
    }
    constexpr Vec& operator*=(T r) {
        x *= r;
        y *= r;
        z *= r;
        return *this;
    }
    constexpr Vec& operator/=(T r) {
        x /= r;
        y /= r;
        z /= r;
        return *this;
    }
    constexpr Vec operator-() const { return Vec(-x, -y, -z); }
    constexpr Vec operator+(const Vec& r) const { return Vec(*this) += r; }
    constexpr Vec operator-(const Vec& r) const { return Vec(*this) -= r; }
    constexpr Vec operator*(T r) const { return Vec(*this) *= r; }
    constexpr Vec operator/(T r) const { return Vec(*this) /= r; }
    friend constexpr Vec operator*(T r, const Vec& v) { return v * r; }
};

// rotation around n=(x,y,z) by theta: cos(theta/2) + (xi+yj+zk) sin(theta/2)
struct Quarternion {
    T x, y, z, w;
    Quarternion() = default;
    constexpr Quarternion(T x, T y, T z, T w) : x(x), y(y), z(z), w(w) {}
    constexpr Quarternion conj() const { return Quarternion(-x, -y, -z, w); }
    constexpr Quarternion& operator+=(const Quarternion& r) {
        x += r.x;
        y += r.y;
        z += r.z;
        w += r.w;
        return *this;
    }
    constexpr Quarternion& operator-=(const Quarternion& r) {
        x -= r.x;
        y -= r.y;
        z -= r.z;
        w -= r.w;
        return *this;
    }
    constexpr Quarternion& operator*=(const Quarternion& r) {
        *this = Quarternion(w * r.x - z * r.y + y * r.z + x * r.w,
                            z * r.x + w * r.y - x * r.z + y * r.w,
                            -y * r.x + x * r.y + w * r.z + z * r.w,
                            -x * r.x - y * r.y - z * r.z + w * r.w);
        return *this;
    }
    constexpr Quarternion& operator*=(T r) {
        x *= r;
        y *= r;
        z *= r;
        w *= r;
        return *this;
    }
    constexpr Quarternion& operator/=(T r) {
        x /= r;
        y /= r;
        z /= r;
        w /= r;
        return *this;
    }
    constexpr Quarternion operator-() const {
        return Quarternion(-x, -y, -z, -w);
    }
    constexpr Quarternion operator+(const Quarternion& r) const {
        return Quarternion(*this) += r;
    }
    constexpr Quarternion operator-(const Quarternion& r) const {
        return Quarternion(*this) -= r;
    }
    constexpr Quarternion operator*(const Quarternion& r) const {
        return Quarternion(*this) *= r;
    }
    constexpr Quarternion operator*(T r) const {
        return Quarternion(*this) *= r;
    }
    constexpr Quarternion operator/(T r) const {
        return Quarternion(*this) /= r;
    }
};

std::istream& operator>>(std::istream& is, Vec& p) {
    T x, y, z;
    is >> x >> y >> z;
    p = {x, y, z};
    return is;
}

std::ostream& operator<<(std::ostream& os, const Vec& p) {
    os << "(" << p.x << ", " << p.y << ", " << p.z << ")";
    return os;
}

T dot(const Vec& a, const Vec& b) { return a.x * b.x + a.y * b.y + a.z * b.z; }

Vec cross(const Vec& a, const Vec& b) {
    return Vec(a.y * b.z - a.z * b.y, a.z * b.x - a.x * b.z,
               a.x * b.y - a.y * b.x);
}

namespace std {
T norm(const Vec& a) { return dot(a, a); }
T abs(const Vec& a) { return std::sqrt(std::norm(a)); }
}  // namespace std

constexpr T eps = 1e-10;
inline bool eq(T a, T b) { return std::abs(a - b) <= eps; }
inline bool eq(Vec a, Vec b) { return std::abs(a - b) <= eps; }
inline bool lt(T a, T b) { return a < b - eps; }
inline bool leq(T a, T b) { return a <= b + eps; }

struct Line {
    Vec p1, p2;
    Line() = default;
    Line(const Vec& p1, const Vec& p2) : p1(p1), p2(p2) {}
    Vec dir() const { return p2 - p1; }
};

struct Segment : Line {
    using Line::Line;
};

struct Plane {
    Vec n, p;
    Plane() = default;
    Plane(const Vec& n, const Vec& p) : n(n), p(p) {}
};

struct Sphere {
    Vec c;
    T r;
    Sphere() = default;
    Sphere(const Vec& c, T r) : c(c), r(r) {}
};

Vec rot(const Vec& v, const Quarternion& q) {
    auto u = q * Quarternion(v.x, v.y, v.z, 0) * q.conj();
    return {u.x, u.y, u.z};
}

// get the rotation that moves a to b
Quarternion get_rotation(const Vec& a, const Vec& b) {
    assert(eq(std::abs(a), 1));
    assert(eq(std::abs(b), 1));

    T theta = std::acos(dot(a, b));
    Vec n = cross(a, b);
    n /= std::abs(n);
    T c = std::cos(theta / 2);
    T s = std::sin(theta / 2);
    return Quarternion(s * n.x, s * n.y, s * n.z, c);
}

bool are_collinear(const Vec& p1, const Vec& p2, const Vec& p3) {
    return eq(std::norm(cross(p2 - p1, p3 - p1)), 0);
}

bool are_coplanar(const Vec& p1, const Vec& p2, const Vec& p3, const Vec& p4) {
    return eq(dot(cross(p2 - p1, p4 - p1), p3 - p1), 0);
}

// --- intersect ---

// 0: skew
// 1: parallel
// 2: intersect
int intersect(const Line& l, const Line& m) {
    if (!are_coplanar(l.p1, l.p2, m.p1, m.p2)) return 0;
    if (eq(std::norm(cross(l.dir(), m.dir())), 0)) return 1;
    return 2;
}

bool intersect(const Plane& pl, const Line& l) {
    return !eq(dot(pl.n, l.dir()), 0);
}

// --- intersection ---

Vec intersection(const Line& l, const Line& m) {
    assert(intersect(l, m) == 2);

    auto r = m.p1 - l.p1;
    auto dlr = dot(l.dir(), r);
    auto dmr = dot(m.dir(), r);
    auto dlm = dot(l.dir(), m.dir());
    auto dll = std::norm(l.dir());
    auto dmm = std::norm(m.dir());

    auto t = (dlr * dmm - dmr * dlm) / (dll * dmm - dlm * dlm);
    return l.p1 + t * l.dir();
}

Vec intersection(const Plane& pl, const Line& l) {
    assert(intersect(pl, l));
    auto dir = l.dir();
    auto d = dot(pl.p - l.p1, pl.n) / dot(dir, pl.n);
    return l.p1 + d * dir;
}
\end{lstlisting}