\section{geometry}

\subsection{Geometry}

\begin{small}
\begin{markdown}
`Vec` は `std::complex<T>` のエイリアスである.

- `geometry.hpp`
    - 基本関数群
    - `T dot(Vec a, Vec b)`
        - 内積を計算する
    - `T cross(Vec a, Vec b)`
        - 外積の $z$ 座標を計算する
    - `Vec rot(Vec a, T ang)`
        - $a$ を角 $ang$ だけ回転させる
    - `Vec perp(Vec a)`
        - $a$ を角 $\pi/2$ だけ回転させる
    - `Vec projection(Line l, Vec p)`
        - 点 $p$ の直線 $l$ 上の射影を求める
    - `Vec reflection(Line l, Vec p)`
        - 点 $p$ の直線 $l$ に関して対称な点を求める
    - `int ccw(Vec a, Vec b, Vec c)`
        - $a,b,c$ が同一直線上にあるなら0, $a \rightarrow b \rightarrow c$ が反時計回りなら1,そうでなければ-1を返す
    - `void sort\_by\_arg(vector<Vec> pts)`
        - 与えられた点を偏角ソートする (ソート順は[この問題](https://judge.yosupo.jp/problem/sort\_points\_by\_argument)に準拠)
        - 時間計算量: $O(n\log n)$
- `intersect.hpp`
    - 交差判定
- `dist.hpp`
    - 距離
- `intersection.hpp`
    - 交点
- `bisector.hpp`
    - 二等分線
- `triangle.hpp`
    - 重心,内心,外心
- `tangent.hpp`
    - 円の接線
- `polygon.hpp`
    - 多角形周り.計算量は明示しない限り $O(n)$
    - `T area(Polygon poly)`
        - 多角形 $poly$ の面積を求める
    - `T is\_convex(Polygon poly)`
        - 多角形 $poly$ が凸か判定する.`poly` は反時計回りに与えられる必要がある
    - `Polygon convex\_cut(Polygon poly, Line l)`
        - 多角形 $poly$ を直線 $l$ で切断する.詳細な仕様は [凸多角形の切断](https://onlinejudge.u-aizu.ac.jp/courses/library/4/CGL/4/CGL\_4\_C) を参照.
    - `Polygon halfplane\_intersection(vector<pair<Vec, Vec>> hps)`
        - 半平面の集合が与えられたとき,それらの共通部分 (凸多角形になる) を返す.半平面は, $\{\boldsymbol{x}\mid(\boldsymbol{x}-\boldsymbol{p})\cdot \boldsymbol{n}\geq 0\}$ で表したときに $(\boldsymbol{n},\boldsymbol{p})$ の形で与える.
        - 時間計算量: $O(n\log n)$

\end{markdown}
\end{small}

\begin{lstlisting}
using T = double;
using Vec = std::complex<T>;

const T PI = std::acos(-1);

constexpr T eps = 1e-10;
inline bool eq(T a, T b) { return std::abs(a - b) <= eps; }
inline bool eq(Vec a, Vec b) { return std::abs(a - b) <= eps; }
inline bool lt(T a, T b) { return a < b - eps; }
inline bool leq(T a, T b) { return a <= b + eps; }

std::istream& operator>>(std::istream& is, Vec& p) {
    T x, y;
    is >> x >> y;
    p = {x, y};
    return is;
}

struct Line {
    Vec p1, p2;
    Line() = default;
    Line(const Vec& p1, const Vec& p2) : p1(p1), p2(p2) {}
    Vec dir() const { return p2 - p1; }
};

struct Segment : Line {
    using Line::Line;
};

struct Circle {
    Vec c;
    T r;
    Circle() = default;
    Circle(const Vec& c, T r) : c(c), r(r) {}
};

using Polygon = std::vector<Vec>;

T dot(const Vec& a, const Vec& b) {
    return (std::conj(a) * b).real();
}

T cross(const Vec& a, const Vec& b) {
    return (std::conj(a) * b).imag();
}

Vec rot(const Vec& a, T ang) {
    return a * Vec(std::cos(ang), std::sin(ang));
}

Vec perp(const Vec& a) {
    return Vec(-a.imag(), a.real());
}

Vec projection(const Line& l, const Vec& p) {
    return l.p1 + dot(p - l.p1, l.dir()) * l.dir() / std::norm(l.dir());
}

Vec reflection(const Line& l, const Vec& p) {
    return T(2) * projection(l, p) - p;
}

// 0: collinear
// 1: counter-clockwise
// -1: clockwise
int ccw(const Vec& a, const Vec& b, const Vec& c) {
    if (eq(cross(b - a, c - a), 0)) return 0;
    if (lt(cross(b - a, c - a), 0)) return -1;
    return 1;
}

void sort_by_arg(std::vector<Vec>& pts) {
    std::sort(pts.begin(), pts.end(), [&](auto& p, auto& q) {
        if ((p.imag() < 0) != (q.imag() < 0)) return (p.imag() < 0);
        if (cross(p, q) == 0) {
            if (p == Vec(0, 0)) return !(q.imag() < 0 || (q.imag() == 0 && q.real() > 0));
            if (q == Vec(0, 0)) return  (p.imag() < 0 || (p.imag() == 0 && p.real() > 0));
            return (p.real() > q.real());
        }
        return (cross(p, q) > 0);
    });
}
\end{lstlisting}

\subsection{intersection.hpp}

\begin{lstlisting}
#include "geometry.hpp"
#include "dist.hpp"

Vec intersection(const Line& l, const Line& m) {
    Vec r = m.p1 - l.p1;
    assert(!eq(cross(l.dir(), m.dir()), 0)); // not parallel
    return l.p1 + cross(m.dir(), r) / cross(m.dir(), l.dir()) * l.dir();
}

std::vector<Vec> intersection(const Circle& c, const Line& l) {
    T d = dist(l, c.c);
    if (lt(c.r, d)) return {};  // no intersection
    Vec e1 = l.dir() / std::abs(l.dir());
    Vec e2 = perp(e1);
    if (ccw(c.c, l.p1, l.p2) == 1) e2 *= -1;
    if (eq(c.r, d)) return {c.c + d*e2};  // tangent
    T t = std::sqrt(c.r*c.r - d*d);
    return {c.c + d*e2 + t*e1, c.c + d*e2 - t*e1};
}

std::vector<Vec> intersection(const Circle& c1, const Circle& c2) {
    T d = std::abs(c1.c - c2.c);
    if (lt(c1.r + c2.r, d)) return {};  // outside
    Vec e1 = (c2.c - c1.c) / std::abs(c2.c - c1.c);
    Vec e2 = perp(e1);
    if (lt(d, std::abs(c2.r - c1.r))) return {};  // contain
    if (eq(d, std::abs(c2.r - c1.r))) return {c1.c + c1.r*e1};  // tangent
    T x = (c1.r*c1.r - c2.r*c2.r + d*d) / (2*d);
    T y = std::sqrt(c1.r*c1.r - x*x);
    return {c1.c + x*e1 + y*e2, c1.c + x*e1 - y*e2};
}

T area_intersection(const Circle& c1, const Circle& c2) {
    T d = std::abs(c2.c - c1.c);
    if (leq(c1.r + c2.r, d)) return 0;  // outside
    if (leq(d, std::abs(c2.r - c1.r))) {  // inside
        T r = std::min(c1.r, c2.r);
        return PI * r * r;
    }
    T ans = 0;
    T a;
    a = std::acos((c1.r*c1.r+d*d-c2.r*c2.r)/(2*c1.r*d));
    ans += c1.r*c1.r*(a - std::sin(a)*std::cos(a));
    a = std::acos((c2.r*c2.r+d*d-c1.r*c1.r)/(2*c2.r*d));
    ans += c2.r*c2.r*(a - std::sin(a)*std::cos(a));
    return ans;
}

\end{lstlisting}

\subsection{dist.hpp}

\begin{lstlisting}
#include "geometry.hpp"
#include "intersect.hpp"

T dist(const Line& l, const Vec& p) {
    return std::abs(cross(p - l.p1, l.dir())) / std::abs(l.dir());
}

T dist(const Segment& s, const Vec& p) {
    if (lt(dot(p - s.p1, s.dir()), 0)) return std::abs(p - s.p1);
    if (lt(dot(p - s.p2, -s.dir()), 0)) return std::abs(p - s.p2);
    return std::abs(cross(p - s.p1, s.dir())) / std::abs(s.dir());
}

T dist(const Segment& s, const Segment& t) {
    if (intersect(s, t)) return T(0);
    return std::min({dist(s, t.p1), dist(s, t.p2), dist(t, s.p1), dist(t, s.p2)});
}

\end{lstlisting}

\subsection{intersect.hpp}

\begin{lstlisting}
#include "geometry.hpp"

bool intersect(const Segment& s, const Vec& p) {
    Vec u = s.p1 - p, v = s.p2 - p;
    return eq(cross(u, v), 0) && leq(dot(u, v), 0);
}

// 0: outside
// 1: on the border
// 2: inside
int intersect(const Polygon& poly, const Vec& p) {
    const int n = poly.size();
    bool in = 0;
    for (int i = 0; i < n; ++i) {
        auto a = poly[i] - p, b = poly[(i+1)%n] - p;
        if (eq(cross(a, b), 0) && (lt(dot(a, b), 0) || eq(dot(a, b), 0))) return 1;
        if (a.imag() > b.imag()) std::swap(a, b);
        if (leq(a.imag(), 0) && lt(0, b.imag()) && lt(cross(a, b), 0)) in ^= 1;
    }
    return in ? 2 : 0;
}

int intersect(const Segment& s, const Segment& t) {
    auto a = s.p1, b = s.p2;
    auto c = t.p1, d = t.p2;
    if (ccw(a, b, c) != ccw(a, b, d) && ccw(c, d, a) != ccw(c, d, b)) return 2;
    if (intersect(s, c) || intersect(s, d) || intersect(t, a) || intersect(t, b)) return 1;
    return 0;
}

// true if they have positive area in common or touch on the border
bool intersect(const Polygon& poly1, const Polygon& poly2) {
    const int n = poly1.size();
    const int m = poly2.size();
    for (int i = 0; i < n; ++i) {
        for (int j = 0; j < m; ++j) {
            if (intersect(Segment(poly1[i], poly1[(i+1)%n]), Segment(poly2[j], poly2[(j+1)%m]))) {
                return true;
            }
        }
    }
    return intersect(poly1, poly2[0]) || intersect(poly2, poly1[0]);
}

// 0: inside
// 1: inscribe
// 2: intersect
// 3: circumscribe
// 4: outside
int intersect(const Circle& c1, const Circle& c2) {
    T d = std::abs(c1.c - c2.c);
    if (lt(d, std::abs(c2.r - c1.r))) return 0;
    if (eq(d, std::abs(c2.r - c1.r))) return 1;
    if (eq(c1.r + c2.r, d)) return 3;
    if (lt(c1.r + c2.r, d)) return 4;
    return 2;
}
\end{lstlisting}

\subsection{tangent.hpp}

\begin{lstlisting}
#include "geometry.hpp"
#include "intersect.hpp"
#include "intersection.hpp"

std::pair<Vec, Vec> tangent_points(const Circle& c, const Vec& p) {
    auto m = (p + c.c) / T(2);
    auto is = intersection(c, Circle(m, std::abs(p - m)));
    return {is[0], is[1]};
}

// for each l, l.p1 is a tangent point of c1
std::vector<Line> common_tangents(Circle c1, Circle c2) {
    assert(!eq(c1.c, c2.c) || !eq(c1.r, c2.r));
    int cnt = intersect(c1, c2);  // number of common tangents
    std::vector<Line> ret;
    if (cnt == 0) {
        return ret;
    }

    // external
    if (eq(c1.r, c2.r)) {
        auto d = c2.c - c1.c;
        Vec e(-d.imag(), d.real());
        e = e / std::abs(e) * c1.r;
        ret.push_back(Line(c1.c + e, c1.c + e + d));
        ret.push_back(Line(c1.c - e, c1.c - e + d));
    } else {
        auto p = (-c2.r*c1.c + c1.r*c2.c) / (c1.r - c2.r);
        if (cnt == 1) {
            Vec q(-p.imag(), p.real());
            return {Line(p, q)};
        } else {
            auto [a, b] = tangent_points(c1, p);
            ret.push_back(Line(a, p));
            ret.push_back(Line(b, p));
        }
    }

    // internal
    auto p = (c2.r*c1.c + c1.r*c2.c) / (c1.r + c2.r);
    if (cnt == 3) {
        Vec q(-p.imag(), p.real());
        ret.push_back(Line(p, q));
    } else if (cnt == 4) {
        auto [a, b] = tangent_points(c1, p);
        ret.push_back(Line(a, p));
        ret.push_back(Line(b, p));
    }

    return ret;
}

\end{lstlisting}

\subsection{polygon.hpp}

\begin{lstlisting}
#pragma once
#include <algorithm>
#include <deque>
#include <utility>
#include <vector>
#include "geometry.hpp"
#include "intersection.hpp"

T area(const Polygon& poly) {
    const int n = poly.size();
    T res = 0;
    for (int i = 0; i < n; ++i) {
        res += cross(poly[i], poly[(i + 1) % n]);
    }
    return std::abs(res) / T(2);
}

bool is_convex(const Polygon& poly) {
    int n = poly.size();
    for (int i = 0; i < n; ++i) {
        if (lt(cross(poly[(i+1)%n] - poly[i], poly[(i+2)%n] - poly[(i+1)%n]), 0)) {
            return false;
        }
    }
    return true;
}

Polygon convex_cut(const Polygon& poly, const Line& l) {
    const int n = poly.size();
    Polygon res;
    for (int i = 0; i < n; ++i) {
        auto p = poly[i], q = poly[(i+1)%n];
        if (ccw(l.p1, l.p2, p) != -1) {
            if (res.empty() || !eq(res.back(), p)) {
                res.push_back(p);
            }
        }
        if (ccw(l.p1, l.p2, p) * ccw(l.p1, l.p2, q) < 0) {
            auto c = intersection(Line(p, q), l);
            if (res.empty() || !eq(res.back(), c)) {
                res.push_back(c);
            }
        }
    }
    return res;
}

Polygon halfplane_intersection(std::vector<std::pair<Vec, Vec>> hps) {
    using Hp = std::pair<Vec, Vec>;  // (normal vector, a point on the border)

    auto intersection = [&](const Hp& l1, const Hp& l2) -> Vec {
        auto d = l2.second - l1.second;
        return l1.second + (dot(d, l2.first) / cross(l1.first, l2.first)) * perp(l1.first);
    };

    // check if the halfplane h contains the point p
    auto contains = [&](const Hp& h, const Vec& p) -> bool {
        return dot(p - h.second, h.first) > 0;
    };

    constexpr T INF = 1e15;
    hps.emplace_back(Vec(1, 0), Vec(-INF, 0));  // -INF <= x
    hps.emplace_back(Vec(-1, 0), Vec(INF, 0));  // x <= INF
    hps.emplace_back(Vec(0, 1), Vec(0, -INF));  // -INF <= y
    hps.emplace_back(Vec(0, -1), Vec(0, INF));  // y <= INF

    std::sort(hps.begin(), hps.end(), [&](const auto& h1, const auto& h2) {
        return std::arg(h1.first) < std::arg(h2.first);
    });

    std::deque<Hp> dq;
    int len = 0;
    for (auto& hp : hps) {
        while (len > 1 && !contains(hp, intersection(dq[len-1], dq[len-2]))) {
            dq.pop_back();
            --len;
        }

        while (len > 1 && !contains(hp, intersection(dq[0], dq[1]))) {
            dq.pop_front();
            --len;
        }

        // parallel
        if (len > 0 && eq(cross(dq[len-1].first, hp.first), 0)) {
            // opposite
            if (lt(dot(dq[len-1].first, hp.first), 0)) {
                return {};
            }
            // same
            if (!contains(hp, dq[len-1].second)) {
                dq.pop_back();
                --len;
            } else continue;
        }

        dq.push_back(hp);
        ++len;
    }

    while (len > 2 && !contains(dq[0], intersection(dq[len-1], dq[len-2]))) {
        dq.pop_back();
        --len;
    }

    while (len > 2 && !contains(dq[len-1], intersection(dq[0], dq[1]))) {
        dq.pop_front();
        --len;
    }

    if (len < 3) return {};

    std::vector<Vec> poly(len);
    for (int i = 0; i < len - 1; ++i) {
        poly[i] = intersection(dq[i], dq[i+1]);
    }
    poly[len-1] = intersection(dq[len-1], dq[0]);
    return poly;
}

\end{lstlisting}

\subsection{triangle.hpp}

\begin{small}
\begin{markdown}

\end{markdown}
\end{small}

\begin{lstlisting}
#include "geometry.hpp"
#include "intersection.hpp"
#include "bisector.hpp"

Vec centroid(const Vec& A, const Vec& B, const Vec& C) {
    assert(ccw(A, B, C) != 0);
    return (A + B + C) / T(3);
}

Vec incenter(const Vec& A, const Vec& B, const Vec& C) {
    assert(ccw(A, B, C) != 0);
    T a = std::abs(B - C);
    T b = std::abs(C - A);
    T c = std::abs(A - B);
    return (a*A + b*B + c*C) / (a + b + c);
}

Vec circumcenter(const Vec& A, const Vec& B, const Vec& C) {
    assert(ccw(A, B, C) != 0);
    return intersection(bisector(Segment(A, B)), bisector(Segment(A, C)));
}
\end{lstlisting}

\subsection{Convex Hull}

\begin{small}
時間計算量: $O(n\log n)$
\end{small}

\begin{lstlisting}
#include "geometry.hpp"

std::vector<Vec> convex_hull(std::vector<Vec>& pts) {
    int n = pts.size();
    if (n == 1) return pts;
    std::sort(pts.begin(), pts.end(), [](const Vec& v1, const Vec& v2) {
        return (v1.imag() != v2.imag()) ? (v1.imag() < v2.imag()) : (v1.real() < v2.real());
    });
    int k = 0; // the number of vertices in the convex hull
    std::vector<Vec> ch(2 * n);
    // right
    for (int i = 0; i < n; ++i) {
        while (k > 1 && lt(cross(ch[k-1] - ch[k-2], pts[i] - ch[k-1]), 0)) --k;
        ch[k++] = pts[i];
    }
    int t = k;
    // left
    for (int i = n - 2; i >= 0; --i) {
        while (k > t && lt(cross(ch[k-1] - ch[k-2], pts[i] - ch[k-1]), 0)) --k;
        ch[k++] = pts[i];
    }
    ch.resize(k - 1);
    return ch;
}
\end{lstlisting}

\subsection{bisector.hpp}

\begin{lstlisting}
#include "geometry.hpp"
#include "intersection.hpp"

Line bisector(const Segment& s) {
    auto m = (s.p1 + s.p2) / T(2);
    return Line(m, m + Vec(-s.dir().imag(), s.dir().real()));
}

std::pair<Line, Line> bisector(const Line& l, const Line& m) {
    // parallel
    if (eq(cross(l.dir(), m.dir()), 0)) {
        auto n = Line(l.p1, l.p1 + perp(l.dir()));
        auto p = intersection(n, m);
        auto m = (l.p1 + p) / T(2);
        return {Line(m, m + l.dir()), Line()};
    }
    auto p = intersection(l, m);
    T ang = (std::arg(l.dir()) + std::arg(m.dir())) / T(2);
    auto b1 = Line(p, p + std::polar(T(1), ang));
    auto b2 = Line(p, p + std::polar(T(1), ang + PI / T(2)));
    return {b1, b2};
}

\end{lstlisting}