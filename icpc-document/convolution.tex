\section{convolution}

\begin{small}
畳み込みライブラリ.演算の種類と用いる変換の対応は以下の通り.

\begin{itemize}
    \item MAX conv: 累積和 (右)
    \item MIN conv: 累積和 (左)
    \item XOR conv: fwht
    \item OR  conv: fzt(true)
    \item AND conv: fzt(false)
    \item GCD conv: divisor\_fzt(false)
    \item LCM conv: divisor\_fzt(true)
    \item $+$ conv: fft
\end{itemize}

計算量はだいたい $O(n\log n)$, 約数FZT/FMTのみ $O(n\log\log n)$
\end{small}

\subsection{Bitwise XOR Convolution}

\begin{lstlisting}
template <typename T>
void fwht(std::vector<T>& a) {
    int n = a.size();
    for (int h = 1; h < n; h <<= 1) {
        for (int i = 0; i < n; i += h << 1) {
            for (int j = i; j < i + h; ++j) {
                T x = a[j];
                T y = a[j | h];
                a[j] = x + y;
                a[j | h] = x - y;
            }
        }
    }
}

template <typename T>
void ifwht(std::vector<T>& a) {
    int n = a.size();
    for (int h = 1; h < n; h <<= 1) {
        for (int i = 0; i < n; i += h << 1) {
            for (int j = i; j < i + h; ++j) {
                T x = a[j];
                T y = a[j | h];
                a[j] = (x + y) / 2;
                a[j | h] = (x - y) / 2;
            }
        }
    }
}
\end{lstlisting}

\subsection{Bitwise AND/OR Convolution}

\begin{lstlisting}
template <typename T>
void fzt(std::vector<T>& a, bool subset) {
    int k = 31 - __builtin_clz(a.size());
    for (int i = 0; i < k; ++i) {
        for (int j = 0; j < (1 << k); ++j) {
            if ((j >> i & 1) == subset) a[j] += a[j ^ (1 << i)];
        }
    }
}

template <typename T>
void fmt(std::vector<T>& a, bool subset) {
    int k = 31 - __builtin_clz(a.size());
    for (int i = 0; i < k; ++i) {
        for (int j = 0; j < (1 << k); ++j) {
            if ((j >> i & 1) == subset) a[j] -= a[j ^ (1 << i)];
        }
    }
}
\end{lstlisting}

\subsection{GCD/LCM Convolution}

\begin{lstlisting}
template <typename T>
void divisor_fzt(std::vector<T>& a, bool subset) {
    int n = a.size();
    std::vector<bool> sieve(n, true);
    for (int p = 2; p < n; ++p) {
        if (!sieve[p]) continue;
        if (subset) {
            for (int k = 1; k * p < n; ++k) {
                sieve[k * p] = false;
                a[k * p] += a[k];
            }
        } else {
            for (int k = (n - 1) / p; k > 0; --k) {
                sieve[k * p] = false;
                a[k] += a[k * p];
            }
        }
    }
}

template <typename T>
void divisor_fmt(std::vector<T>& a, bool subset) {
    int n = a.size();
    std::vector<bool> sieve(n, true);
    for (int p = 2; p < n; ++p) {
        if (!sieve[p]) continue;
        if (subset) {
            for (int k = (n - 1) / p; k > 0; --k) {
                sieve[k * p] = false;
                a[k * p] -= a[k];
            }
        } else {
            for (int k = 1; k * p < n; ++k) {
                sieve[k * p] = false;
                a[k] -= a[k * p];
            }
        }
    }
}
\end{lstlisting}

\subsection{Fast Fourier Transform}

\begin{small}
NTT では, \texttt{omega} を \texttt{primitive\_root.pow((mod - 1) / m)} にする.mod と原子根の組は
(167772161, 3), (469762049, 3), (754974721, 11), (998244353, 3), (1224736769, 3)

\end{small}

\begin{lstlisting}
constexpr double PI = 3.14159265358979323846;

void fft(std::vector<std::complex<double>>& a) {
    int n = a.size();
    for (int m = n; m > 1; m >>= 1) {
        double ang = 2.0 * PI / m;
        std::complex<double> omega(cos(ang), sin(ang));
        for (int s = 0; s < n / m; ++s) {
            std::complex<double> w(1, 0);
            for (int i = 0; i < m / 2; ++i) {
                auto l = a[s * m + i];
                auto r = a[s * m + i + m / 2];
                a[s * m + i] = l + r;
                a[s * m + i + m / 2] = (l - r) * w;
                w *= omega;
            }
        }
    }
}

void ifft(std::vector<std::complex<double>>& a) {
    int n = a.size();
    for (int m = 2; m <= n; m <<= 1) {
        double ang = -2.0 * PI / m;
        std::complex<double> omega(cos(ang), sin(ang));
        for (int s = 0; s < n / m; ++s) {
            std::complex<double> w(1, 0);
            for (int i = 0; i < m / 2; ++i) {
                auto l = a[s * m + i];
                auto r = a[s * m + i + m / 2] * w;
                a[s * m + i] = l + r;
                a[s * m + i + m / 2] = l - r;
                w *= omega;
            }
        }
    }
}

template <typename T>
std::vector<double> convolution(const std::vector<T>& a, const std::vector<T>& b) {
    int size = a.size() + b.size() - 1;
    int n = 1;
    while (n < size) n <<= 1;
    std::vector<std::complex<double>> na(a.begin(), a.end()), nb(b.begin(), b.end());
    na.resize(n);
    nb.resize(n);
    fft(na);
    fft(nb);
    for (int i = 0; i < n; ++i) na[i] *= nb[i];
    ifft(na);
    std::vector<double> ret(size);
    for (int i = 0; i < size; ++i) ret[i] = na[i].real() / n;
    return ret;
}
\end{lstlisting}
