\section{tree}

\subsection{Cartesian Tree}

\begin{small}
\begin{markdown}
**Description**

Cartesian tree は,数列から定まる二分木で,以下の条件を満たすものである.
- 各頂点の重みは,そのどの子の重みよりも小さい
- 木の in-order traversal がもとの数列と一致する

**Operations**

- `vector<int> cartesian\_tree(vector<int> a)`
    - 数列 $a$ から定まる Cartesian tree を返す.それぞれの頂点の親のラベルを返す.根の親は `-1` とする.
    - 時間計算量: $O(n)$

\end{markdown}
\end{small}

\begin{lstlisting}
template <typename T>
std::vector<int> cartesian_tree(const std::vector<T>& a) {
    int n = a.size();
    std::vector<int> par(n, -1);
    std::stack<int> st;
    for (int i = 0; i < n; ++i) {
        int j = -1;
        while (!st.empty() && a[st.top()] >= a[i]) {
            j = st.top();
            st.pop();
        }
        if (!st.empty()) {
            par[i] = st.top();
        }
        if (j != -1) {
            par[j] = i;
        }
        st.push(i);
    }
    return par;
}
\end{lstlisting}

\subsection{Lowest Common Ancestor}

\begin{small}
\begin{markdown}
- `int query(int u, int v)`
    - 頂点 $u$ と頂点 $v$ の最小共通祖先を返す
    - 時間計算量: $O(\log n)$
- `int dist(int u, int v)`
    - $uv$ 間の距離を計算する
    - 時間計算量: $O(\log n)$
- `int parent(int v, int k)`
    - 頂点 $v$ の $k$ 個上の頂点を求める
- `int jump(int u, int v, int k)`
    - $uv$ パス上の $k$ 番目の頂点を返す.$k=0$ のとき $u$ を,$k>dist(u,v)$ のとき $-1$ を返す.
    - 時間計算量: $O(\log n)$

\end{markdown}
\end{small}

\begin{lstlisting}
class LCA {
public:
    LCA() = default;
    LCA(const std::vector<std::vector<int>>& G, int root) : G(G), LOG(32 - __builtin_clz(G.size())), depth(G.size()) {
        int V = G.size();
        table.assign(LOG, std::vector<int>(V, -1));

        dfs(root, -1, 0);

        for (int k = 0; k < LOG - 1; ++k) {
            for (int v = 0; v < V; ++v) {
                if (table[k][v] >= 0) {
                    table[k + 1][v] = table[k][table[k][v]];
                }
            }
        }
    }

    int query(int u, int v) const {
        if (depth[u] > depth[v]) std::swap(u, v);

        // go up to the same depth
        for (int k = 0; k < LOG; ++k) {
            if ((depth[v] - depth[u]) >> k & 1) {
                v = table[k][v];
            }
        }
        if (u == v) return u;

        for (int k = LOG - 1; k >= 0; --k) {
            if (table[k][u] != table[k][v]) {
                u = table[k][u];
                v = table[k][v];
            }
        }
        return table[0][u];
    }

    int dist(int u, int v) const {
        return depth[u] + depth[v] - 2 * depth[query(u, v)];
    }

    int parent(int v, int k) const {
        for (int i = LOG - 1; i >= 0; --i) {
            if (k >= (1 << i)) {
                v = table[i][v];
                k -= 1 << i;
            }
        }
        return v;
    }

    int jump(int u, int v, int k) const {
        int l = query(u, v);
        int du = depth[u] - depth[l];
        int dv = depth[v] - depth[l];
        if (du + dv < k) return -1;
        if (k < du) return parent(u, k);
        return parent(v, du + dv - k);
    }


protected:
    const std::vector<std::vector<int>>& G;
    const int LOG;
    std::vector<std::vector<int>> table;
    std::vector<int> depth;

    void dfs(int v, int p, int d) {
        table[0][v] = p;
        depth[v] = d;
        for (int c : G[v]) {
            if (c != p) dfs(c, v, d + 1);
        }
    }
};
\end{lstlisting}

\subsection{Tree Isomorphism}

\begin{small}
\begin{markdown}
- `TreeHasher`
    - 確率的だが高速.ハッシュを用いている.
    - `long long hash\_all(vector<vector<int>> G, int root)`
        - 木 $G$ の隣接リストが与えられたとき,$G$ のハッシュを計算する.`root` が与えられた場合はそれを根とする.そうでない場合は,直径の中心を根とする.直径の中心が複数ある場合は,ハッシュ値が小さくなる方を返す
        - 時間計算量: $O(n)$
    - `vector<long long> hash\_subtrees(vector<vector<int>> G, int root)`
        - 木 $G$ の隣接リストが与えられたとき,$G$ の各部分木のハッシュを計算する.
        - 時間計算量: $O(n)$
- `TreeEncoder`
    - 決定的だが低速.AHU algorithm を用いている.
    - `vector<int> encode(vector<vector<int>> G, int root)`
        - 木 $G$ の隣接リストが与えられたとき,$G$ の各部分木のラベルを計算する.
        - 時間計算量: $O(n\log n)$


\end{markdown}
\end{small}

\begin{lstlisting}
#include "tree_diameter.cpp"

class TreeHasher {
public:
    TreeHasher() : rng(rd()), rand(1, mod-1) {}

    long long hash_all(const std::vector<std::vector<int>>& G, int root = -1) {
        long long res;
        if (root == -1) {
            auto [d, path] = tree_diameter(G);
            res = dfs_all(G, path[d / 2], -1).first;
            if (d % 2 == 1) {
                res = std::min(res, dfs_all(G, path[d / 2 + 1], -1).first);
            }
        } else {
            res = dfs_all(G, root, -1).first;
        }
        return res;
    }

    std::vector<long long> hash_subtrees(const std::vector<std::vector<int>>& G, int root) {
        std::vector<long long> hash(G.size());
        dfs_subtrees(G, hash, root, -1);
        return hash;
    }

private:
    static constexpr long long mod = (1LL << 61) - 1;

    static inline long long add(long long a, long long b) {
        if ((a += b) >= mod) a -= mod;
        return a;
    }

    static inline long long mul(long long a, long long b) {
        __int128_t c = (__int128_t) a * b;
        return add(c >> 61, c & mod);
    }

    std::random_device rd;
    std::mt19937_64 rng;
    std::uniform_int_distribution<long long> rand;
    std::vector<long long> R;

    std::pair<long long, int> dfs_all(const std::vector<std::vector<int>>& G, int v, int p) {
        int maxd = 0;
        std::vector<long long> hash;
        for (int c : G[v]) {
            if (c != p) {
                auto [h, d] = dfs_all(G, c, v);
                maxd = std::max(maxd, d + 1);
                hash.push_back(h);
            }
        }
        if ((int) R.size() == maxd) {
            R.push_back(rand(rng));
        }
        long long res = 1;
        for (auto h : hash) {
            res = mul(res, add(R[maxd], h));
        }
        return {res, maxd};
    }

    int dfs_subtrees(const std::vector<std::vector<int>>& G, std::vector<long long>& hash, int v, int p) {
        int maxd = 0;
        for (int c : G[v]) {
            if (c != p) {
                maxd = std::max(maxd, dfs_subtrees(G, hash, c, v) + 1);
            }
        }
        if ((int) R.size() == maxd) {
            R.push_back(rand(rng));
        }
        long long res = 1;
        for (int c : G[v]) {
            if (c != p) {
                res = mul(res, add(R[maxd], hash[c]));
            }
        }
        hash[v] = res;
        return maxd;
    }
};


class TreeEncoder {
public:
    TreeEncoder() {
        mp[{}] = 0;
    }

    std::vector<int> encode(const std::vector<std::vector<int>>& G, int root) {
        std::vector<int> val(G.size());
        dfs(G, val, root, -1);
        return val;
    }

private:
    std::map<std::vector<int>, int> mp;
    std::vector<long long> R;

    void dfs(const std::vector<std::vector<int>>& G, std::vector<int>& val, int v, int p) {
        std::vector<int> ch;
        for (int c : G[v]) {
            if (c != p) {
                dfs(G, val, c, v);
                ch.push_back(val[c]);
            }
        }
        std::sort(ch.begin(), ch.end());
        if (!mp.count(ch)) {
            mp[ch] = mp.size();
        }
        val[v] = mp[ch];
    }
};
\end{lstlisting}

\subsection{Centroid Decomposition}

\begin{small}
\begin{markdown}
- `tuple<vector<int>, vector<int>, vector<int>> centroid\_decomposition(vector<vector<int>> G)`
  - 木 $G$ の隣接リストが与えられたとき,3つ組 `(level, sz, par)` を返す.
    - `level`: $G$ を重心分解したときの各頂点のレベル (何回目の分割でそれが重心となるか)
    - `sz`: 各頂点が重心となるときにそれが含まれる部分木のサイズ
    - `par`: 各頂点が重心となる直前に属していた部分木の重心
  - 時間計算量: $O(n\log n)$

\end{markdown}
\end{small}

\begin{lstlisting}
std::tuple<std::vector<int>, std::vector<int>, std::vector<int>> centroid_decomposition(const std::vector<std::vector<int>>& G) {
    int N = G.size();
    std::vector<int> sz(N), level(N, -1), sz_comp(N), par(N);

    auto dfs_size = [&](auto& dfs_size, int v, int p) -> int {
        sz[v] = 1;
        for (int c : G[v]) {
            if (c != p && level[c] == -1) sz[v] += dfs_size(dfs_size, c, v);
        }
        return sz[v];
    };

    auto dfs_centroid = [&](auto& dfs_centroid, int v, int p, int n) -> int {
        for (int c : G[v]) {
            if (c != p && level[c] == -1 && sz[c] > n / 2) return dfs_centroid(dfs_centroid, c, v, n);
        }
        return v;
    };

    auto decompose = [&](auto& decompose, int v, int k, int p) -> void {
        int n = dfs_size(dfs_size, v, -1);
        int s = dfs_centroid(dfs_centroid, v, -1, n);
        level[s] = k;
        sz_comp[s] = n;
        par[s] = p;
        for (int c : G[s]) {
            if (level[c] == -1) decompose(decompose, c, k + 1, s);
        }
    };

    decompose(decompose, 0, 0, -1);
    return {level, sz_comp, par};
}
\end{lstlisting}

\subsection{Heavy-Light Decomposition}

\begin{small}
\begin{markdown}
`update` および `fold` の時間計算量を $f(n)$ とする

- `HLD(vector<vector<int>> G, bool edge)`
    - 木 `G` を HL 分解する.`edge == true` ならクエリは辺に対して実行される.
    - 時間計算量: $O(n)$
- `void update(int v, T x, F update)`
    - 頂点 $v$ に対して `update(x)` を実行する
    - 時間計算量: $O(f(n) \log n)$
- `void update\_edge(int u, int v, T x, F update)`
    - 辺 $(u, v)$ に対して `update(x)` を実行する
    - 時間計算量: $O(f(n) \log n)$
- `void update(int u, int v, T x, F update)`
    - $uv$ パス上の頂点/辺に対して `update(x)` を実行する.
    - 時間計算量: $O(f(n) \log n)$
- `T path\_fold(int u, int v, F fold)`
    - $uv$ パス上の頂点/辺に対して `fold()` を実行する.
    - 時間計算量: $O(f(n) \log n)$
- `T path\_fold(int u, int v, F fold, Flip flip)`
    - $uv$ パス上の頂点/辺に対して `fold()` を実行する.値が非可換なら,左から積をとったときと右から積をとったときの値を入れ替える `flip` 関数を与える必要がある.
    - 時間計算量: $O(f(n) \log n)$
- `T subtree\_fold(int v, F fold)`
    - 頂点 $v$ を根とする部分木の頂点/辺に対して `fold()` を実行する.
    - 時間計算量: $O(f(n))$
- `int lca(int u, int v)`
    - 頂点 $u$ と頂点 $v$ の最小共通祖先を返す
    - 時間計算量: $O(\log n)$
- `int dist(int u, int v)`
    - $uv$ 間の距離を計算する
    - 時間計算量: $O(\log n)$

\end{markdown}
\end{small}

\begin{lstlisting}
template <typename M>
class HLD {
    using T = typename M::T;

public:
    HLD() = default;
    HLD(const std::vector<std::vector<int>>& G, bool edge)
        : G(G), size(G.size()), depth(G.size()), par(G.size(), -1),
          in(G.size()), out(G.size()), head(G.size()), heavy(G.size(), -1), edge(edge) {
        dfs(0);
        decompose(0, 0);
    }

    template <typename F>
    void update(int v, const T& x, const F& f) const {
        f(in[v], x);
    }

    template <typename F>
    void update_edge(int u, int v, const T& x, const F& f) const {
        if (in[u] > in[v]) std::swap(u, v);
        f(in[v], x);
    }

    template <typename E, typename F>
    void update(int u, int v, const E& x, const F& f) const {
        while (head[u] != head[v]) {
            if (in[head[u]] > in[head[v]]) std::swap(u, v);
            f(in[head[v]], in[v] + 1, x);
            v = par[head[v]];
        }
        if (in[u] > in[v]) std::swap(u, v);
        f(in[u] + edge, in[v] + 1, x);
    }

    template <typename F, typename Flip>
    T path_fold(int u, int v, const F& f, const Flip& flip) const {
        bool flipped = false;
        T resu = M::id(), resv = M::id();
        while (head[u] != head[v]) {
            if (in[head[u]] > in[head[v]]) {
                std::swap(u, v);
                std::swap(resu, resv);
                flipped ^= true;
            }
            T val = f(in[head[v]], in[v] + 1);
            resv = M::op(val, resv);
            v = par[head[v]];
        }
        if (in[u] > in[v]) {
            std::swap(u, v);
            std::swap(resu, resv);
            flipped ^= true;
        }
        T val = f(in[u] + edge, in[v] + 1);
        resv = M::op(val, resv);
        resv = M::op(flip(resu), resv);
        if (flipped) {
            resv = flip(resv);
        }
        return resv;
    }

    template <typename F>
    T path_fold(int u, int v, const F& f) const {
        return path_fold(u, v, f, [&](auto& v) { return v; });
    }

    template <typename F>
    T subtree_fold(int v, const F& f) const {
        return f(in[v] + edge, out[v]);
    }

    int lca(int u, int v) const {
        while (true) {
            if (in[u] > in[v]) std::swap(u, v);
            if (head[u] == head[v]) return u;
            v = par[head[v]];
        }
    }

    int dist(int u, int v) const {
        return depth[u] + depth[v] - 2 * depth[lca(u, v)];
    }

private:
    std::vector<std::vector<int>> G;
    std::vector<int> size, depth, par, in, out, head, heavy;
    bool edge;
    int cur_pos = 0;

    void dfs(int v) {
        size[v] = 1;
        int max_size = 0;
        for (int c : G[v]) {
            if (c == par[v]) continue;
            par[c] = v;
            depth[c] = depth[v] + 1;
            dfs(c);
            size[v] += size[c];
            if (size[c] > max_size) {
                max_size = size[c];
                heavy[v] = c;
            }
        }
    }

    void decompose(int v, int h) {
        head[v] = h;
        in[v] = cur_pos++;
        if (heavy[v] != -1) decompose(heavy[v], h);
        for (int c : G[v]) {
            if (c != par[v] && c != heavy[v]) decompose(c, c);
        }
        out[v] = cur_pos;
    }
};

\end{lstlisting}

\subsection{Diameter of a Tree}

\begin{lstlisting}
#include "../graph/edge.cpp"

std::pair<int, std::vector<int>> tree_diameter(const std::vector<std::vector<int>>& G) {
    std::vector<int> to(G.size());

    auto dfs = [&](const auto& dfs, int v, int p) -> std::pair<int, int> {
        std::pair<int, int> ret(0, v);
        for (int c : G[v]) {
            if (c == p) continue;
            auto weight = dfs(dfs, c, v);
            ++weight.first;
            if (ret < weight) {
                ret = weight;
                to[v] = c;
            }
        }
        return ret;
    };

    auto p = dfs(dfs, 0, -1);
    auto q = dfs(dfs, p.second, -1);
    std::vector<int> path;
    int v = p.second;
    while (v != q.second) {
        path.push_back(v);
        v = to[v];
    }
    path.push_back(v);
    return {q.first, path};
}

template <typename T>
std::pair<T, std::vector<int>> tree_diameter(const std::vector<std::vector<Edge<T>>>& G) {
    std::vector<int> to(G.size());

    auto dfs = [&](const auto& dfs, int v, int p) -> std::pair<T, int> {
        std::pair<T, int> ret(0, v);
        for (auto& e : G[v]) {
            if (e.to == p) continue;
            auto weight = dfs(dfs, e.to, v);
            weight.first += e.weight;
            if (ret < weight) {
                ret = weight;
                to[v] = e.to;
            }
        }
        return ret;
    };

    auto p = dfs(dfs, 0, -1);
    auto q = dfs(dfs, p.second, -1);
    std::vector<int> path;
    int v = p.second;
    while (v != q.second) {
        path.push_back(v);
        v = to[v];
    }
    path.push_back(v);
    return {q.first, path};
}
\end{lstlisting}

\subsection{Rerooting}

\begin{small}
\begin{markdown}
DPは $dp\_v = g(f(dp\_{c\_1}, e\_1) * \dots * f(dp\_{c\_k}, e\_k), v)$ という形の遷移で表されるとする.

**Template Parameters**

- `M`
    - 可換モノイド
- `Cost`
    - 辺のコストの型
- `T apply\_edge(T a, int s, int t, Cost c)`
    - 遷移の $f$
- `T apply\_vertex(T x, int v)`
    - 遷移の $g$

**Operations**

- `Rerooting(int n)`
    - 頂点数 $n$ で木を初期化する
    - 時間計算量: $O(n)$
- `void add\_edge(int u, int v, Cost c)`
    - 頂点 $uv$ 間にコスト $c$ の辺を張る
    - 時間計算量: $O(1)$
- `vector<T> run()`
    - 各頂点を根としたときの木DPの値を求める
    - 時間計算量: $O(n)$

\end{markdown}
\end{small}

\begin{lstlisting}
template <typename M, typename Cost,
          typename M::T (*apply_edge)(typename M::T, int, int, Cost),
          typename M::T (*apply_vertex)(typename M::T, int)>
class Rerooting {
    using T = typename M::T;

   public:
    explicit Rerooting(int n) : G(n) {}

    void add_edge(int u, int v, Cost c) {
        G[u].emplace_back(v, c);
        G[v].emplace_back(u, c);
    }

    std::vector<T> run() {
        dp_sub.resize(G.size(), M::id());
        dp_all.resize(G.size());
        dfs_sub(0, -1);
        dfs_all(0, -1, M::id());
        return dp_all;
    }

   private:
    std::vector<std::vector<std::pair<int, Cost>>> G;
    std::vector<T> dp_sub, dp_all;

    void dfs_sub(int v, int p) {
        for (auto [c, cost] : G[v]) {
            if (c == p) continue;
            dfs_sub(c, v);
            dp_sub[v] = M::op(dp_sub[v], apply_edge(dp_sub[c], v, c, cost));
        }
        dp_sub[v] = apply_vertex(dp_sub[v], v);
    }

    void dfs_all(int v, int p, const T& val) {
        std::vector<T> ds = {val};
        for (auto [c, cost] : G[v]) {
            if (c == p) continue;
            ds.push_back(apply_edge(dp_sub[c], v, c, cost));
        }
        int n = ds.size();
        std::vector<T> head(n + 1, M::id()), tail(n + 1, M::id());
        for (int i = 0; i < n; ++i) head[i + 1] = M::op(head[i], ds[i]);
        for (int i = n - 1; i >= 0; --i) tail[i] = M::op(ds[i], tail[i + 1]);
        dp_all[v] = apply_vertex(head[n], v);
        int k = 1;
        for (auto [c, cost] : G[v]) {
            if (c == p) continue;
            dfs_all(c, v,
                    apply_edge(apply_vertex(M::op(head[k], tail[k + 1]), v), c,
                               v, cost));
            ++k;
        }
    }
};

\end{lstlisting}